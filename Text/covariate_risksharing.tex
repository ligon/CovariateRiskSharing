\documentclass[AER]{AEA}

\makeatletter
% Hook into \appendix to clear the redundant reference prefix
\g@addto@macro\appendix{%
  \renewcommand{\p@subsection}{}% Remove the "C." prefix
}
\makeatother

\usepackage[utf8]{inputenc}
% \usepackage[T1]{fontenc}
\usepackage{lmodern}
\usepackage{amsmath}
%\usepackage{amsthm}
\usepackage{amssymb}
\usepackage{capt-of}
\usepackage{hyperref}
            \usepackage{nicefrac}
\newcommand{\Var}{\ensuremath{\mbox{Var}}}
\newcommand{\R}{\ensuremath{\mathbb{R}}}
\usepackage{dsfont}\newcommand{\one}{\ensuremath{\mathds{1}}}
\usepackage{bm}\renewcommand{\vec}[1]{\bm{#1}}
\usepackage{econometrics}
\newcommand{\Eq}[1]{(\ref{eq:#1})}
\newcommand{\argmax}{\operatornamewithlimits{arg\,max}}

\draftSpacing{1.5}

%% ox-latex features:
%   !announce-start, bibliography-biblatex, !guess-pollyglossia, !guess-babel,
%   !guess-inputenc, caption, maths, image, bibliography-biblatex-resources,
%   !announce-end.

%\usepackage[style=apa,backend=biber,sortcites=ynt]{biblatex}
\usepackage{natbib}
\let\autocite\citep
\newcommand{\textcite}[2][]{\citet[#1]{#2}} % Ligon (2025)
\newcommand{\parencite}[2][]{\citep[#1]{#2}}
\bibliographystyle{aea}


\usepackage{capt-of}

\usepackage{amsmath}
\usepackage{amssymb}

\usepackage{graphicx}

% end precompiled preamble
\ifcsname endofdump\endcsname\endofdump\fi

%\addbibresource{~/bibtex/main.bib}

%% end ox-latex features

\usepackage{tabularx}\usepackage[online]{threeparttable}\usepackage{booktabs}
\usepackage{dcolumn}
\usepackage{float}
%\usepackage{fullpage}
\usepackage{setspace}\onehalfspacing
\usepackage[margin=1in]{geometry}
\usepackage{pdflscape}
\setcounter{secnumdepth}{3}
%\address{University of California, Berkeley}
\newtheorem{proposition}{Proposition} \newcommand{\Prop}[1]{Proposition \ref{prop:#1}}
\newtheorem{theorem}{Theorem} \newcommand{\Thm}[1]{Theorem \ref{thm:#1}}
\newtheorem{lemma}{Lemma} \newcommand{\Lem}[1]{Lemma \ref{lem:#1}}
\newtheorem{remark}{Remark} \newcommand{\Rem}[1]{Remark \ref{rem:#1}}
\newtheorem{corollary}{Corollary} \newcommand{\Cor}[1]{Corollary \ref{cor:#1}}
% \theoremstyle{definition}
\newtheorem{definition}{Definition} \newcommand{\Defn}[1]{Definition \ref{defn:#1}}
%\theoremstyle{plain}
\newtheorem{assumption}{Assumption}\newcommand{\Ass}[1]{Assumption \ref{ass:#1}}
\newtheorem{condition}{Condition} \newcommand{\Cond}[1]{Condition \ref{cond:#1}}
\newcommand{\Fig}[1]{Figure \ref{fig:#1}} \newcommand{\Tab}[1]{Table \ref{tab:#1}}
\newcommand{\Sec}[1]{Section \ref{sec:#1}}
\newcommand{\App}[1]{Appendix \ref{app:#1}}
%\addbibresource{main.bib}
\newcommand{\ts}[1]{\textsuperscript{#1}}\newcommand{\hd}[1]{\multicolumn{1}{c}{\text{#1}}}
\usepackage{longtable}\setlength\LTcapwidth{\textwidth}
%\usepackage[notref]{showkeys}\renewcommand*\showkeyslabelformat[1]{ \fbox{\parbox[t]{2cm}{\raggedright\normalfont\tiny\ttfamily\hyphenchar\font=`_#1}}}
\setcounter{secnumdepth}{2}
\author{Ethan Ligon}
\date{March 30, 2025}
\title{Risk Sharing Tests and Covariate Shocks}
\hypersetup{
 pdfauthor={Ethan Ligon},
 pdftitle={},
 pdfkeywords={},
 pdfsubject={},
 pdfcreator={},
 pdflang={English}}
\begin{document}


\title{Risk Sharing Tests and Covariate Shocks: Drought, Floods, and Pests in Uganda}
\shortTitle{Covariate Shocks}
\author{Ethan Ligon\thanks{University of California, Berkeley (ligon\@berkeley.edu).  Acknowledgements}} 


\date{\today}

\begin{abstract}
Efficient risk-sharing implies a simple factor structure for marginal utilities of expenditure (MUEs): Pareto weights divided by a common price.  The standard approach infers MUEs from total expenditures, implicitly assuming homothetic preferences, unitary income elasticities, and identical price elasticities.  Risk-sharing tests using total expenditures work for idiosyncratic shocks (budgets change, but not prices), but not ``covariate'' shocks (prices change).  We describe all preferences which permit one to infer MUEs from expenditures, and estimate nonhomothetic MUEs to test whether covariate shocks are shared efficiently in Uganda.  This delivers sensible results; the standard approach suggests that droughts, floods, and pests are beneficial.
\end{abstract}

\maketitle
\thispagestyle{empty}


\Keywords{Risk sharing, Two-way fixed effects, nonhomothetic demands, Uganda, Marginal utility of expenditures, Aggregate shocks, Droughts, Floods, Pests}


%\thanks{I wish to thank participants of seminars at Berkeley ARE, the World Bank, IFPRI, and colleagues who provided comments based on presentations at ThReD and Georgetown for useful comments.  I wish to further thank the LSMS-ISA team, particularly Gero Carletto and Phillip Wollburg, for assistance with data.}

Most households in Uganda are engaged in agriculture, both for income and subsistence.  Smallholder farming is a very risky activity, with both crops and livestock vulnerable to a litany of shocks: disease, drought or poorly timed rainfall, floods, agricultural pests, shocks (illness or death) to household labor supply, fire, landslides, thefts, and sometimes violence.  Beyond these immediate threats to livelihood, the prices faced by farmers for both inputs and outputs are highly variable and uncertain.  If these poor farmers had to bear such shocks alone, mortality would be much higher than it is.  Instead, it has long been understood that where they can, farmers may cope by \emph{sharing} certain risks with family and friends.

As a formal way of thinking about sharing risk, it is well known that in a setting with risk-averse households and uncertainty, an efficient allocation will eliminate any ``idiosyncratic'' risk, in the sense that given this allocation the ratio of any two households' marginal utilities will vary only in fixed proportion, regardless of the realization of the uncertain state.  Suppose for example that Farmer A is unlucky and a hailstorm damages his rice crop.  Aggregate resources are reduced by the size of the damage, even if only A's crops were affected.  So Farmer A's marginal utility of rice will increase, but if efficiency prevails, then everyone else's marginal utility of rice will increase proportionally, even if the shock was ``idiosyncratic'' in that it only affected the crops of one farmer.

The risk-sharing problem is often framed in terms of a planning problem, with different households assigned some Pareto weights \emph{ex ante} and the planner choosing allocations such that ratios of households' marginal utilities of different goods (e.g., A's marginal utility of rice to B's marginal utility of rice) are equal to the ratio of their Pareto weights.  But the problem immediately decentralizes in the sense that we can describe it in terms of prices and expenditures.  In particular, we can say that an efficient allocation (a) keeps ratios of any two households' marginal utilities of expenditures (MUEs) constant across date-states; and (b) that given these constant ratios,  allocations (or household expenditures) depend \emph{only} on common prices, faced by all the households efficiently sharing risk.\footnote{There are many different ways in which such efficient allocations could be implemented.  Some of these would involve formal markets for contingent claims \autocite{arrow-debreu54} or repeated exchange of some set of securities \autocite{arrow64,arrow-hahn83} or forward markets \autocite{townsend78}.  These market may not even need to be complete \autocite{levine-zame02}.  Neither must they be formal; various forms of informal exchange \autocite{platteau-abraham87} or reciprocity \autocite{cashdan85} may well constitute a system of informal insurance that approaches full efficiency, even in the complete absence of formal markets or legally enforceable contracts \autocite{fafchamps92,ligon-etal02}.}

\textcite{townsend94} wondered whether the kinds of sharing observed in low income countries might be an economically efficient way of dealing with risk within villages, and pioneered methods for testing this ``efficient risk-sharing'' hypothesis in the development literature.  The general approach involves regressing a measure of the (log) marginal utility of expenditures (MUE) on the (log) average of households' MUEs, allowing for household-specific intercepts.  The idea is to directly test the proportionality of marginal utilities of expenditures implied by efficiency.   And in particular this proportionality implies a testable exclusion restriction: the event of the ``idiosyncratic'' shock should \emph{not} affect individual MUEs after controlling for its effect on the average MUE.  Tests along these lines have become the stuff of textbooks \autocite{ray98d,bardhan-udry99} and have been conducted in settings all over the world, in hundreds of different studies.\footnote{Some recent prominent examples in the development literature include \textcite{angelucci-degiorgi09,mazzocco-saini12,chiappori-etal14,jack-suri14,munshi-rosenzweig16,samphantharak-townsend18,santaeulalia-zheng18,kinnan22}, while earlier studies are covered in surveys such as \textcite{alderman-paxson94,townsend95,morduch95,dercon05,attanasio-weber10}.  The technique we attribute to Townsend arguably started in macro or international \autocite{mace91,cochrane91,obstfeld93} and is still used in those fields \autocite{attanasio-davis96,degiorgi-etal20,lee-lee25}.  Though testing for risk-sharing across households is the classic approach, a similar setup is used in other settings, for example to test for sharing across people or generations in the unitary household model \citep[e.g.,][]{altonji-etal92,hayashi-etal96,dercon-krishnan00a,lise-yamada19,theloudis-etal25}.\label{fn:examples}}

However, it should not escape notice that important shocks may not be idiosyncratic.  A drought that affects the household's crop production is very likely to also affect the crops of  others.   If \emph{everyone} is similarly affected by a shock sharing won't help, but this would be unusual: even a drought which affects all households in an area will affect differently situated households differently.  Different crops will be more or less affected by the drought, for example, so a household growing maize may be more affected than a household growing cassava.

If risk is shared efficiently, then how would we expect a covariate shock to affect outcomes?   The central precept that marginal utilities of expenditure should vary in proportion will still hold, so if we were to regress log MUEs on the village average, allowing for household fixed effects, then efficient risk-sharing implies that the exclusion restriction should still hold.   However, if the shock affects enough production in the village, then we might also expect the shock to have an effect on MUEs because of its effect on local \emph{prices}.  Viewed from this perspective, a covariate shock is one that affects prices, while an idiosyncratic shock does not.\footnote{Appendix \ref{app:shocks_and_farmgate_prices} documents such a positive association between reported shocks and farmgate prices in Uganda.\label{fn:shocks_and_prices}}

We would like to be able to test whether the effects of \emph{covariate} shocks such as droughts, floods, or agricultural pests which have heterogeneous effects on many households are also mitigated by sharing, but the Townsend test is generally understood to be poorly suited to testing the sharing of covariate risks.  This paper explores the reasons for this.  The issue turns out to hinge on the construction of MUEs and assumptions made about household preferences.

\label{r2:homotheticity}
Existing approaches to measuring MUEs assume some form of homothetic preferences, for which MUEs can be written as a function of total expenditures while changes in prices can be accounted for via a single price index.  The most commonly encountered case is preferences exhibiting constant relative risk aversion (CRRA).  We demonstrate in Section \ref{sec:risksharing} that in the CRRA case this leads to a test involving a two-way fixed effects (TWFE) regression of log total expenditures on a  set of household fixed effects and a set of market-time effects.  If preferences are correctly specified, then the market-time effects will capture the effects of changes in prices.

But CRRA preferences are very restrictive: among other things all goods must have unitary income elasticities, and expenditure shares for different consumption goods will not vary with income.  \textcite{engel1857}  and a host of more recent evidence emphatically asserts \autocite[e.g.,][]{jensen-miller08} that utility functions are \emph{not} homothetic,\footnote{We present additional evidence on this point in Appendix \ref{sec:org8b65133}.} and so the usual risk-sharing regression is mis-specified.  No single price index can correctly account for the effects of price changes on demands or MUEs, and by its very nature a covariate shock will tend to change prices.  In terms of the usual risk-sharing test, without homothetic preferences prices affect total expenditures in a non-separable way, so that the market-time effects of the TWFE regression cannot control for variation in prices, and the disturbance term in the TWFE regression will depend on prices.  Using our data from Uganda, we show that this mis-specification can lead to surprising results: in the usual specification covariate shocks such as droughts, floods, or pests appear to lead to welfare improvements (Section \ref{sec:tests})!


\Sec{risksharing} discusses full risk-sharing in the general case.  Here the theory is remarkably clear on two basic points, both of which I believe to be novel.  First, one cannot sum up expenditures on goods that have different income elasticities and use this sum to construct MUEs.  Second, in order to use time effects to handle the effects of (possibly unobserved) prices on item-level expenditures, we must be able to express the system of expenditures as an additively separable function of the MUE and prices.

We want a test of risk-sharing which works for both idiosyncratic and covariate shocks.  A great strength of the Townsend approach is that to construct the CRRA measure of MUEs one needs data only on expenditures, data which is widely collected in household surveys in low income countries.  We restrict ourselves to methods which can be implemented using the same data, and undertake the task of finding the broader class of preferences which permits us to infer MUEs using only data on expenditures.

Inferring MUEs from expenditures requires some form of separability between MUEs and prices.  In \Sec{inferring_mues} we show that the requirement that (some transformation of) item-level expenditures have this separability property (along with some standard regularity conditions) is equivalent to the utility function taking one of two particular semi-parametric forms, but only one of these forms can be estimated using linear methods.  Importantly, this class of utility functions nests the homothetic forms previously used in tests of risk sharing.  This class of utility functions then dictates the form of demands and the risk-sharing test.  Each of these two points is independent of actual risk-sharing arrangements.

We next turn to empirics: Given the form demands must take if we're to control for prices using time effects, how can we construct estimates of MUEs?   The equations for log expenditures take a form first explored by \textcite{joreskog-goldberger75} that \textcite{hansen22:econometrics} calls a factor model with additional regressors.  \Sec{estimating_w} discusses the estimation of MUEs using expenditure data.  \Sec{data} describes the Ugandan data we use to estimate MUEs, while in \Sec{estimation} we use those data to construct estimates of MUEs, which are the natural objects to use in a test of risk-sharing.

We follow this in \Sec{effect_of_shocks} with a brief narrative discussion of the effects of various covariate shocks on outcomes in Uganda.  We provide direct evidence on the nonhomotheticity of preferences; establish that droughts, floods, or pests are indeed covariate; document reports that these shocks negatively affect production and income; find that the incidence of these shocks tends to increase farmgate prices and change relative market prices for consumption goods; and finally show that covariate shocks have impacts on households' food diets, with decreases in dietary diversity and intake of certain nutrients such as vitamin B-12.

In \Sec{tests} we use these to conduct the risk-sharing test, and demonstrate that this more general demand system delivers much more sensible results in a test of risk-sharing than does the standard CRRA approach.  Consistent with the CRRA approach we find that idiosyncratic shocks are shared, though not fully.  But the CRRA approach suggests that the impacts of droughts, floods, and pests are not shared but \emph{improve} the well-being of those who experience them.  In contrast, our approach indicates that the impact of these covariate shocks is largely shared via the common channel of changes in local prices, reducing welfare for everyone.

Finally, our construction of MUEs is independent of the risk-sharing hypothesis, and so these objects could be used in many other  settings, or in the estimation and testing of the many dynamic models which put structure on the evolution of MUEs over time.  In \Sec{mues_everywhere} we offer some thoughts and suggestions about ways in which these might proceed.
\section{Risk sharing (The General Case)}
\label{sec:risksharing}
Suppose that preferences are ``regular'' (i.e., can be represented by an increasing, concave, continuously differentiable utility function).  More restrictively, assume that preferences are von Neumann-Morgenstern and intertemporally separable.

Following \textcite{townsend94}, consider the problem facing a social planner in an environment with uncertainty; at date \(t\) state \(s\in\{1,2,\dots,S\}\) is realized with probability \(\mbox{Pr}_t(s)\).  The planner maximizes a weighted sum of households' utilities, with the (Pareto) weight \(\theta_i\) associated with household \(i\)'s utility.  Utility at date \(t\) is discounted by some \(\beta_t\in(0,1)\).  At each date-state \((t,s)\) the planner allocates a consumption \emph{budget} to each household, but has to respect the aggregate resource constraint that the sum of all expenditures must be less than some given quantity \(\bar{x}_t(s)\).  For household \(i\) at date \(t\) in state \(s\) let the budget allocated be \(x_{it}(s)\).  Given this budget and taking prices as given the household then solves the usual consumer problem.

In general, indirect utility within the period will depend not only on the size of  the budget, but also a complete vector of prices \(p_t(s)\) and household characteristics \(z_{it}\).  Then the planner's intertemporal problem can be written
\begin{equation}
\label{eq:planning_problem}
\max_{x_{it}(s)} \sum_i\theta_i\sum_t\beta_t\sum_s\mbox{Pr}_t(s) V(x_{it}(s),p_t(s);z_{it}(s))
\end{equation}
subject to the aggregate budget constraint \(\sum_i x_{it}(s)=\bar{x}_t(s)\) for all \(t,s\).

Let \(\nu_t(s)=\mu_t(s)\beta_t\mbox{Pr}_t(s)\) be the multiplier on the aggregate resource constraint at time \(t\) in state \(s\).  First order conditions associated with the assignment of \(x_{it}(s)\) are
\[
      \theta_i\frac{\partial V}{\partial x}(x_{it}(s),p_t(s);z_{it}(s)) = \theta_i\lambda(x_{it}(s),p_t(s);z_{it}(s))=\mu_t(s) \qquad\text{for all $(i,t,s)$,}
\]
where \(\lambda(x,p,z)\) is a function that can be interpreted as the household's marginal utility of expenditures (MUE), and where \(\mu_t(s)\) is the shadow price associated with the aggregate budget within date-state \((t,s)\).  Taking logarithms and rearranging,
\begin{equation}
\label{eq:my_regression}
  \log\lambda_{it}(s) = \log\mu_t(s) - \log\theta_i.
\end{equation}
Equation \Eq{my_regression} is the hallmark of full risk-sharing, expressing the simple factor structure of optimal allocations.  It also immediately lends itself to testing: The right hand side can be estimated using panel data with two-way fixed effects (time and household), assuming only that prices are common.

Everything up to this point is standard and implied by Pareto optimality, provided only that agents are risk averse and have well-behaved preferences which are separable across dates and states.

However, before taking the predictions of full risk sharing to data one must take a stand on how to construct the MUE function \(\lambda(x,p,z)\).  Townsend adopted a representation of momentary household utility which depends only on the consumption aggregate, normalized by a price index and a scalar function of household characteristics; this is equivalent to assuming a homothetic utility function \autocite{blackorby-donaldson88}.   The empirical literature has mostly followed his example.  Typical risk-sharing tests following Townsend assume homothetic or (less often) quasi-homothetic \citep[e.g.,][]{ogaki-zhang01,zhang-ogaki04} preferences, for example assuming the household indirect utility function takes the ``Constant Relative Risk Aversion'' (CRRA) form
\[
   V(x,p,z)= \frac{(x/(\pi(p)g(z)))^{1-\gamma}-1}{1-\gamma},
\]
where \(\pi(p)\) is a scalar price index, and \(g(z)\) is a scalar function mapping household characteristics into  ``adult equivalents''.\footnote{Townsend (1994) actually works principally with exponential or CARA utility, with \(V(x,p,z)=-\frac{1}{\sigma}\exp[-\sigma\left(x/\pi(p) - g(z)\right)]\) which delivers a regression specified in levels rather than logs of total expenditures, but the subsequent literature has generally adopted the CRRA specification.}   The household's MUE is given by
\[
  \lambda(x,p,z) =\frac{x^{-\gamma}}{\left(\pi(p)g(z)\right)^{1-\gamma}}.
\]
Substituting this expression into \Eq{my_regression} and re-arranging yields an estimating equation of the form exploited by Townsend,
\begin{equation}
\label{eq:townsend}
    \log x_{it} = \frac{1}{\gamma}\log\theta_i - \left[\frac{1}{\gamma}\log\mu_t  + \frac{1-\gamma}{\gamma}\log\pi(p_t)\right] - \frac{1-\gamma}{\gamma}\log g(z_{it}),
\end{equation}
which \textcite{deaton90b} noticed could be implemented as a two-way fixed effects regression.  So estimate this by regressing the log of the consumption aggregate on household fixed effects (which identify \(\gamma^{-1}\log\theta_i\)) and time (or perhaps village-time) effects (which identify the term in square brackets involving only prices), and some known function \(g\) of observed household characteristics.

From the comparison of the two equations \Eq{my_regression} and \Eq{townsend} three important points emerge.  First, that the Townsend risk-sharing regression is a special case of the more general \Eq{my_regression}.  Second, that the marginal utility of expenditures \(\lambda(x,p,z)\) already automatically incorporates information on household characteristics that may affect demand, and does so much more flexibly than does the \(g(z)\) function that appears in the Townsend approach \autocite[c.f.,][]{lewbel10}, as \(g(z)\) cannot vary across different goods.   Third, in general \(\lambda\) also depends on the entire vector of prices \(p\), while in the CRRA case prices affect total expenditures only via a single scalar price index \(\pi(p)\).

Townsend-style risk-sharing tests are generally implemented by adding some measure of a ``shock'' to \Eq{townsend}, and testing the exclusion restriction (idiosyncratic shocks shouldn't affect MUEs).  But even if perfect risk-sharing doesn't hold (we can think of this as the Pareto weights \(\theta_i\) varying with the state), total expenditures in this framework depend only on Pareto weights (reflecting households' relative wealths), on prices \(p\) (capturing aggregate shocks to demand and supply), and on characteristics \(z\) (which may function as demand ``shifters'' \label{remark:shifters}).   ``Shocks'' can affect total expenditures only via one of these three channels.

Under a maintained hypothesis of full risk-sharing, idiosyncratic income variation will be insured, but variation in prices  will still affect expenditures, implying a regression of the form
\begin{equation}
\label{eq:risksharing}
\log\lambda_{it} = \log\mu_t - \log\theta_i + \delta\mbox{Shock}_{it} + e_{it}.
\end{equation}
When we observe \(\lambda_{it}\), we can implement this regression simply by adding two-way fixed effects---time effects account for \(\log\mu_t\) and prices, while household fixed effects account for \(\log\theta_i\).  This leaves a disturbance term \(e_{it}\) which can be interpreted as either (or both of) measurement error in the dependent variable, or the effects of time-varying unobserved household characteristics.  In either case full insurance implies the exclusion restriction \(\delta=0\). (Here we also assume that the shock doesn't affect unobserved household characteristics.)  Note that in this case full insurance implies \(\delta=0\) regardless of whether the shock is idiosyncratic (doesn't affect prices) or covariate (affects prices).

\label{par:interpret_delta}
What about the case in which we observe \(\lambda_{it}\), but there's not full insurance?    In this framework a covariate shock can affect households in two different ways.  First, via a common price channel---a drought may induce a local supply shock, changing prices (and MUEs) for all households.  Second, via an idiosyncratic income channel---farmers who grow, e.g., more or less maize will have their incomes affected to a greater or lesser extent, and if their incomes are not fully insured this will induce idiosyncratic variation in their MUEs.  This correlation implies \(\delta\neq 0\), in which case we would reject full insurance, and interpret \(\delta\) as the average of the idiosyncratic response.

Now, consider the special case of CRRA preferences, for which MUEs can be expressed in the separable form \(\log\lambda(x,p,z)=-\gamma\log\left(\frac{x}{\pi(p)g(z)}\right) -\log\pi(p) - \log g(z)\).  Adding a ``shock'' to the right-hand side of \Eq{townsend} yields the CRRA risk-sharing regression
\begin{equation}
\label{eq:crra_risksharing}
\log x_{it} = \frac{1}{\gamma}\log\theta_{i} - \frac{1}{\gamma}\left[\log\mu_t + (1-\gamma)\log\pi(p_t)\right] + \delta\mbox{Shock}_{it} - \frac{1-\gamma}{\gamma}g(z_{it}) + \frac{1}{\gamma}e_{it},
\end{equation}
where now the \emph{joint} hypothesis of full risk-sharing and CRRA preferences implies \(\delta=0\). Suppose that there's full risk-sharing but preferences are \emph{not} CRRA.   If the shock is idiosyncratic (and so doesn't affect prices, and also doesn't affect characteristics \(z\)) then we would still expect \(\delta=0\).  But if the shock affects prices then the exclusion restriction fails, because in this case the single price index \(\pi(p)\) can't account for the effects of changes in prices on the composition of expenditures.  It follows that the disturbance term \(e_{it}\) must be a function of those prices. \label{ref:joint_test}

More particularly, in the face of increased prices expenditures on inelastic goods such as food will \emph{increase}.  And since most of the expenditure items we have household-level data on are different sorts of food, we might expect a measure of total food expenditures to be positively correlated with ``covariate'' negative shocks which increase local prices, such as drought, floods, pests, or changes in prices for agricultural inputs.
To put a finer point on it, if the measure of \(x_{it}\) is not expenditures on \emph{all} non-durable goods and services, but is expenditures on a subset of goods which have inelastic demands, then by definition expenditures on that subset will tend to increase with increases in prices of those goods.  Per \textcite{engel1857} the subset ``food'' would be a good example, with more recent evidence including \Citet{mckenzie03,thomas-frankenberg07}.  And in this case we would predict that any shock that causes increases in food prices will be associated with \(\delta>0\).

So, to sum up: Covariate shocks affect prices and hence expenditures in ways that aren't properly accounted for in a specification of the risk-sharing regression that assumes homothetic preferences.  Thus, even if there \emph{is} full insurance covariate shocks may be correlated with total expenditure.  Further, if the constructed consumption aggregate excludes some elastic goods in particular, then we would expect the correlation between covariate shocks and the constructed consumption aggregate to be positive.
\section{Inferring MUEs from Expenditure Data}
\label{sec:inferring_mues}
The problem we've identified is that the usual risk-sharing regression provides a joint test of full risk-sharing \emph{and} CRRA utility.   Assuming CRRA utility allows us to use nothing more than panel data on expenditures (and perhaps household demographics) to construct risk-sharing tests based on panel estimation of two-way fixed effects.   In practice these are highly desirable properties.  Are there nonhomothetic utility structures which could more flexibly account for demand responses to changes in prices while at the same time preserving these desirable properties?\footnote{A complementary approach taken by \textcite{atkin-etal24} exploits nonhomothetic utility structures which can flexibly respond to changes in relative prices, but delivers money-metric utility changes, rather than MUEs, and so is inappropriate for measuring risk-sharing.\label{fn:atkin-etal24}}   In this section we establish that the answer is ``yes'', and obtain the complete class of  demand systems which (i) are implied by some regular utility function; and (ii) can be constructed using (just) data on consumption expenditures (and perhaps household demographics).
\subsection{Preliminaries}
\label{sec:preliminaries}
Here we begin to develop the basic theory necessary to describe the class of demands that interests us.  As a first step we set aside any explicit consideration of household characteristics, and focus on demand systems that are functions of just budget and prices.  There's no loss of generality in this. With time-separable von-Neumann-Morgenstern preferences we can simply describe utility functions and demand implicitly conditioning on those characteristics; we reintroduce those characteristics as demand-shifters when we turn to empirics.

Let \(\mathcal{U}_n\) be the set of strictly increasing, strictly concave, twice-continuously differentiable functions mapping \(\R^n_+\) into \(\R\), and call \(\mathcal{U}_n\) the set of \emph{regular} utility functions over \(\R^n_+\).

For a household with a utility function \(U\in\mathcal{U}_n\) with a total budget \(\bar x>0\) facing prices \(p\in\R^n_+\), a Lagrangian formulation of the \emph{consumer's problem} is to solve \(\max_{c\in\R^n_+} U(c) + \lambda(\bar x-p\cdot c)\), with \(\lambda\) the Lagrange multiplier or MUE that we would like to obtain for our risk-sharing test.

We can express a solution to the consumer's problem in terms of demand functions which depend on prices and \(\lambda\), a form of demands advocated by Ragnar Frisch.   So we say that Frischian demands map the product of positive quantity \(\lambda\) and \(n\) prices into \(n\) quantities demanded.  We say that

\begin{definition}
An \(n\)-vector of Frischian demands \(f(p,\lambda)\) is \emph{rationalized} by \(U\)
if there exists a \(U\in\mathcal{U}_n\) such that
  \begin{equation}
\label{eq:foc}
   u_j(f(p,\lambda)) := \frac{\partial U}{\partial c_j}(f(p,\lambda)) = p_j\lambda
\end{equation}
  for all \(p\) in any open subset of \(\R^n_+\) and \(\lambda>0\)  for
\(j=1,\dots,n\).
\label{defn:valid}
\end{definition}
Similarly, we say a given \(f\) is \emph{rationalizable} if there exists a \(U\in\mathcal{U}_n\) which rationalizes \(f\).

\Defn{valid} basically\footnote{The definition doesn't seem to allow for important real-world cases, such as two goods which are perfect substitutes (the rationalizing utility function wouldn't be strictly concave).  However, extending the consumer's problem to allow for household production allows us to cover this and a broad class of other cases \autocite{ligon20estimating}.\label{fn:substitutes}} requires that demands be interior solutions to the problem of maximizing some regular utility function subject to a budget constraint.  If a consumer has a utility function \(U\), and solutions to that consumer's problem are characterized by the first order conditions \Eq{foc}, then these demands will also be solutions to this consumer's problem.

From the form of the right-hand side of \Eq{foc} it's apparent that Frischian demands must be homogeneous of degree zero in \((p,1/\lambda)\).  It follows that we can simplify the expression of Frischian demands, as we have \(f(p,\lambda)=f(p\lambda,1)\), which can then simply be written as \(f(p\lambda)\).  This same point follows from the observation that at the optimum the vector of marginal utilities  \(u\) can be inverted to yield \(f(p,\lambda)=u^{-1}(p\lambda)\), so that we know we can write Frischian demands as simply \(f(p\lambda)\).
\subsection{MUEs for risk-sharing tests}
\label{sec:orgc5a8686}
The problem: we observe consumption expenditures \(\{x_j\}\) for some (but perhaps not all) goods \(j\).  From these data we wish to infer values for MUEs; further, we want to be able to use these MUEs in a risk-sharing test that can be implemented using two-way fixed effects.

The fundamental risk-sharing equation is
\[
      \log\lambda = \log\mu(p) - \log\theta,
\]
but we don't directly observe \(\lambda\), only expenditures.   So if we're to preserve the risk-sharing test we need to be able to write some transformation of expenditures as an additively separable function of prices and \(\log \lambda\).   Note that CRRA utility does exactly this as expenditures for good \(j\) in the CRRA case satisfy \(-\gamma\log x_j = \log\lambda + (1-\gamma)\log p_j\), or (summing over goods) \(\log x = \frac{-1}{\gamma}\log\lambda + \log\pi(p)\) for some linearly homogeneous price index \(\pi(p)\).  However, other more general preferences will also work.  What are these preferences?

In general, for any good \(j\) we need functions \((\phi_j,a_j)\) such that
\begin{equation}
\label{eq:separable}
\phi_j(x_j) = \log\lambda + a_j(p).
\end{equation}
If there exist functions \((\phi_j,a_j)\) satisfying \Eq{separable}, then we can substitute into the fundamental risk-sharing equation, obtaining
\[
     \phi_j(x_j) = [\log\mu(p) + a_j(p)] - \log\theta.
\]
Since the term in brackets varies only with prices this can serve as the basis of the kind of risk-sharing test that we're after, with time-effects identifying \(\log\mu(p) + a_j(p)\) and household fixed effects identifying \(\log \theta\).  The key property is \Eq{separable}; this is a special case of a more general property we'll call \(\lambda\)-separability.\footnote{This property generalizes what \textcite{browning-etal85} calls ``Case 2'' demands, discussed in detail in \textcite{ligon16:lambda_separable_frisch}.}

\begin{definition}
The Frischian expenditures on good \(j\), \(x_j(p,\lambda)\equiv p_jf_j(p\lambda)\) are \(\lambda\)-\emph{separable} if there exist functions \((\phi_j,a_j,b_j)\) such that
  \begin{equation}
\label{eq:xlseparable}
    \phi_j(x_j(p,\lambda)) = a_j(p) +  b_j(\lambda),
\end{equation}
with \(\phi_j\) continuously differentiable and \(a_j\) either varies with \(p\) or is equal to zero.
\label{defn:xlseparable}
\end{definition}

Note that while rationalizability is a property of the entire system of demands and expenditures, (\(\lambda\)-) separability is a property of a particular good.  In particular it's possible that some but not all demands or expenditures are (\(\lambda\)-) separable.
\subsection{Demands and utilities when expenditures are \(\lambda\)-separable}
\label{sec:org2c7c64f}

Exploiting the fact that expenditures must be linearly homogeneous, it turns out that one can write any rationalizable \(\lambda\)-separable expenditures in the form
  \[
    k(p+\lambda) = g(\lambda)\ell(p) + h(p),
 \]
which is called the \emph{generalized Pexider} equation (Appendix \ref{sec:org4896922}).  This gives us a single functional equation in two variables, which can be solved for the four functions \(g\), \(h\), \(k\), and \(\ell\).  Exploiting this allows us to describe all rationalizable demands and utilities when expenditures are \(\lambda\)-separable:

\begin{theorem}
If demands are rationalizable (\Defn{valid}) and expenditures on good  \(j\) are \(\lambda\)-separable (\Defn{xlseparable}) with \(\phi_j\) increasing; \(a_j(p)\) either
non-constant or zero, and continuous at a point; and with \(b_j\)
continuous at a point, then transformation functions \(\phi_j\), Frischian
demands \(f_j\) and rationalizing marginal utility \(u_j\) must belong to
one of the following two cases of demands for positive constants \(\alpha_j\) and
\(\beta_j\), and constants \(\sigma_j\):
\begin{enumerate}
\item (Constant Frisch Elasticity): \(\phi_j(x_j)=\log(x_j)\);
\(f_j(p,\lambda)=(\alpha_j/(\lambda p_j))^{\beta_j}\); and  \(u_j(c)=\alpha_jc_j^{-1/\beta_j}\).
\item (Generalized Stone-Geary): \(\phi_j(x_j)=x_j^{\beta_j}\);
\(f_j(p,\lambda)=[(\alpha_j/(\lambda p_j))^{\beta_j} +
     \sigma_j]^{1/\beta_j}\); and
       \(u_j(c)=\alpha_j\left(c_j^{\beta_j} - \sigma_j\right)^{-1/\beta_j}\).
\end{enumerate}
\label{thm:xtaxonomy}
\end{theorem}

\begin{proof}
See Appendix \ref{sec:org4896922}.
\end{proof}
\subsubsection*{Rationalizing Utility Functions}
\label{sec:orgd00be0f}

The labels of the different cases in \Thm{xtaxonomy} indicate names for the rationalizing utility function \(U\) having marginal utilities \(u_j(c)\); an example of ``Constant Frisch Elasticity'' (CFE) utility can be written as \(U(c)=\sum_{j=1}^n\alpha_j\beta_j\frac{c_j^{1-1/\beta_j}-1}{\beta_j-1}\).\footnote{\Citet{ligon20estimating} gave this name to a general form of this utility function, but special cases include the ``direct addilog'' of \textcite{houthakker60} or the ``constant relative income elasticity'' form of \textcite{caron-etal14}.}  The CFE system generalizes the Constant Elasticity of Substitution (CES) system (take \(\beta_j=\beta\)), of which the Cobb-Douglas system is a limiting case (take \(\beta\rightarrow 1\), applying L'Hôpital's rule).   Both the CES and Cobb-Douglas cases are homothetic, and consistent with the CRRA indirect utility function. Finally, the ``Generalized Stone-Geary'' case gives what is, to the best of my knowledge, a marginal utility function which has not previously appeared in the literature.  This case gives demands which are not linear in parameters, which may limit its usefulness in applied empirical work.  However, when \(\beta_j=1\) one obtains the quasi-homothetic Stone-Geary utility function, which suggests that it could be used to explore the behavior of Engel curves, perhaps exploiting a Box-Cox approach to estimation.
\subsection{Estimating \(\log \lambda\)}
\label{sec:estimating_w}
\Thm{xtaxonomy} gives us Frischian quantities demanded as a function of prices and MUEs, but our present motivation for using these demands is to obtain estimates of MUEs from these demand systems using only expenditures.   Other practical concerns are also important: we do not wish to claim that we observe expenditures for all goods, just a subset; the expenditure system should allow for rich and unknown patterns of substitution; the demand system should allow for expenditures on some goods to be zero.

\label{r1:zeros}
To this end, imagine extending the demand system by imagining that demands  of \Thm{xtaxonomy} to be for goods which may be purchased at a price \(p_j\), or it may be produced at a unit cost \(P_j(p)\) by using a vector of inputs having prices \(p\), with \(P_j(p)\) linearly  homogeneous.  The expenditure systems of the theorem are \(\lambda\)-separable by construction, so we know that after transforming expenditures by taking logarithms (CFE) or raising expenditures on \(j\) to a power \(\beta_j\) (Generalized Stone-Geary) we will obtain an expression which is additively separable in \(\lambda\) and \(p\).  An exact parametric form governing the relation between \(\lambda\) and expenditures is given by the theorem, but in the extended demand system prices can influence expenditures via any linearly homogeneous function \(\tilde{a}_j(p)\).

Accordingly, using \Thm{xtaxonomy}, the condition that expenditures be \(\lambda\)-separable implies that expenditures must take one of two forms:\footnote{Further, \textcite{ligon20estimating} demonstrates that expenditures on an input \(k\) used to produce a top-level good \(j\) with CFE demand will itself take the form \(\log x_k = \log a_{jk}(p) - \beta_j\log\lambda\).}
\begin{enumerate}
\item \(\log x_j = \log \tilde{a}_j(p) - \beta_j\log\lambda\); or
\item \(x_j = \left[\left(\frac{\alpha_j}{\lambda}\right)^{\beta_j} + a_j(p)\right]^{1/\beta_j}\).
\end{enumerate}

The second form (generalized Stone-Geary) does not easily allow us to estimate log MUEs using data on expenditures.  But the first does, using standard methods of estimation with panel data.   Here, instead of the panel dimensions varying over households and time periods, they vary over households and items of consumption expenditure.   In particular, for notational convenience let \(w_{it}=-\log\lambda_{it}\), and index goods by \(j\) and households by \(i\).  Then we have for \emph{each} period \(t\)
\begin{equation}
\label{eq:estimating}
     \log x_{it}^j =  a_j(p_t) + g_j(z_{it}) + \beta_j w_{it} + \epsilon_{it}^j,
\end{equation}
where \(a_j(p_t)\) measures the effect of common time \(t\) prices \(p_t\) on expenditures, where \(z_{it}\) are observable household characteristics, such as household size and composition, where \(\beta_j\) gives us (minus) the elasticity of expenditures on good \(j\) with respect to MUE, and where \(\epsilon_{it}^j\) can be regarded as measurement error in expenditures, or perhaps the effects of unobserved household characteristics on those expenditures.   As prices are assumed to be common, we can account for the term \(a_j(p_t)\)  using good-time effects (or good-time-market effects, if it is thought that prices vary across spatially distinct market regions, in which case \(t\) should be thought of as indexing different market-periods).\footnote{In principle rather than assuming that all households face the same prices and using latent variables one could use data on the prices (or marginal costs, if pricing was non-linear) to estimate the \(a_j(p_t)\) if it was thought that the law of one price was unlikely to hold.  But this would be a high-dimensional problem, and one of the chief attractions of the CFE demand system is that it can be estimated \emph{without} observing prices.\label{fn:different_prices}  Further, some heterogeneity in unobserved prices can be accommodated.  If, for example, differently situated households faced proportional iceberg costs for any goods acquired this would lead to observationally equivalent outcome of facing the same prices, but having a larger MUE.}

We call the system \Eq{estimating} the \emph{Constant Frisch Elasticity} (CFE) expenditure system, as it is an example of the systems considered by \textcite{frisch59} with its chief distinguishing characteristic the fact that the Frischian elasticities (the \(\beta_j\)) of expenditures with respect to MUE are constant.

Different strategies may be employed to estimate the functions \(g_j\); our preference here is perhaps the simplest, which is to simply assume a linear form \(g_j(z)=\gamma_j'z\) for some vector of parameters \(\gamma_j\) which may vary across goods.  Then estimation proceeds in two steps.  First, estimate the regression
\begin{equation}
\label{eq:estimating1}
     \log x_{it}^j =  a^j_t + \gamma_j'z_{it}  + e^j_{it},
\end{equation}
with \(a^j_t\) a set of good-time-market effects, and \(e^j_{it}\) a composite error term, since from \Eq{estimating} we have \(e^j_{it} = - \beta_j w_{it} + \epsilon_{it}^j\).

We consider three different possibilities for our second step.  First, summing the error terms \(e^j_{it}\) over different goods \(j\) for each household \(i\) at \(t\) we obtain
\[
    w_{it} = \frac{\sum_j e^j_{it}}{\sum_j\beta_j} - \frac{\sum_j \epsilon^j_{it}}{\sum_j\beta_j}.
\]
We can assume that \(\frac{1}{J}\sum_j\beta_j = 1\) without loss of generality, as this simply amounts to choosing the scale on which we measure \(w_{it}\).  Then a natural analog estimator of \(w_{it}\) is simply the average of the estimated residuals \(\hat{e}_{it}^j\) from \Eq{estimating1}, or
\[
   \tilde{w}_{it} = \frac{1}{J}\sum_j \hat{e}^j_{it}.
\]
This will be an unbiased estimator under standard conditions (principally that \((w_{it},z_{it})\) are orthogonal to the disturbance \(\epsilon^j_{it}\) in \Eq{estimating}).  Call this the ``average'' estimator of \(w_{it}\).

Second, if the variance of \(\epsilon^j_{it}\) varies across goods, then (using the usual logic of generalized least squares) we can construct a more efficient estimator.  Let \(\sigma_j^2\) denote the variance of the error for the \(j\)th good.  In this heteroskedastic case the  ``weighted average'' \(\hat{w}_{it} = \frac{1}{J}\sum_j \hat{e}^j_{it}/\sigma_j\)will be a more efficient estimator of \(w_{it}\).  The standard deviation of the estimated residuals \(\hat{e}_{it}^2\) computed from the sample of all households in all periods is a feasible estimator for \(\sigma_j\).

But if accounting for variance improves efficiency, then the GLS logic encourages us to further consider covariance.  This leads to a model similar to that of \textcite{joreskog-goldberger75} and our third estimator.  Rather than averaging, this ``factor analysis'' approach focuses on the sample covariance matrix of the residuals.  From the definition of \(e_{it}^j\) this covariance matrix takes the form \(\Sigma = \mbox{Var}(w)\beta\beta' + \Psi\), so this approach involves estimating the rank one matrix \(\beta\beta'\) given the sample covariance matrix \(\hat{\Sigma}\).  One can then obtain estimates of the \(w_{it}\) via a regression which includes the \(\hat{\beta}\) coefficients as generated regressors, estimating
\[
    \log x_{it}^j =  a^j_t + \gamma_j'z_{it}  + \hat\beta_j w_{it} + \tilde{\epsilon}^j_{it}.
\]
This approach has several benefits, including increased efficiency, but it also yields estimates of the \(\beta_j\) as well as the MUEs, and  because it allows us to compare \(w\)s over time.  This comes at the cost of requiring some additional structure on the covariance matrix \(\Psi\) of the \(\epsilon_{it}^j\) disturbances.  The factor analysis estimator is described in greater detail in Appendix \ref{app:details_of_estimation}.\footnote{Additional details of assumptions and methods are given in \textcite{ligon20estimating}, while code to compute estimates is provided by \textcite{ligon25:cfe_code}.}

Here we adopt the factor analysis approach estimator for our main results.  This allows us to present estimates of \(\beta\), and of MUEs that  are comparable over time.   But as it happens, in our present application the choice of estimator turns out to matter very little, as the correlations between the factor analysis estimates of the \(w_{it}\) and the average estimates are \(0.97\), while the correlation between the two averaging estimators is \(0.99\).  Almost none of the results we present on MUEs vary importantly with the choice of estimator.\footnote{The only exceptions are Figures \ref{fig:w_by_year_joyplot} and  \ref{fig:covariate_shocks_by_month_one_way}, which take as input \(w\)s which are comparable over time.}
\section{Data from Uganda}
\label{sec:data}
A goal of this paper is to provide tools to estimate MUEs with only modest data requirements.  In particular, we've devised methods which allow us to construct estimates of MUEs from a rich class of nonhomothetic preferences using nothing more than data on expenditures and basic household demographics.  These are the same data that are often used to construct MUEs for homothetic preferences, typically some measure of total consumption expenditures per capita or adult equivalent.
\subsection{Sample Data From Uganda}
\label{sec:orgf108cdb}
\nocite{ubosvarious}
Household MUEs can be estimated using a single cross-section, but our test of risk-sharing requires panel data on households over time.  We use panel survey data from Uganda, covering eight waves which span the years 2005--2020.  These are the Ugandan National Panel Surveys, and are conducted by the Ugandan Bureau of Statistics part of the World Bank's \emph{Living Standards Measurement Surveys} (LSMSs), which have now been conducted in many countries across many years \autocite{deaton97}.  Waves were not always collected at regular intervals, however, and only a subset of households appear multiple times.  In particular, while we observe usable data from 5601 distinct households over this period, 1007 of these households appear only once, and so contribute nothing of value to the risk sharing test (though data on these households' consumption expenditures is still useful for estimating the CFE demand system).  \Tab{attrition} describes the number of households observed in each round (after dropping those households missing the data to necessary to estimate the demand system),  and how many households observed in a particular round also turn up in another.  For example, looking at the first row of the table, we have data for 2911 households in 2005, and of these 1187 are also observed in 2019-20.

\begin{table}[htbp]
\caption{\label{tab:attrition}Households' Attrition in the Ugandan LSMS.  Includes only households with data necessary to estimate demand system.}
\centering
\begin{tabular}{rlllrrrrr}
Round & 2005 & 2009 & 2010 & 2011 & 2013 & 2015 & 2018 & 2019\\
\hline
2005-06 & 2911 & 2313 & 2094 & 2139 & 1444 & 1354 & 1252 & 1187\\
2009-10 & --- & 2760 & 2295 & 2336 & 1532 & 1441 & 1319 & 1253\\
2010-11 & --- & --- & 2457 & 2206 & 1444 & 1355 & 1251 & 1188\\
2011-12 & --- & --- & --- & 2687 & 1575 & 1477 & 1364 & 1291\\
2013-14 & --- & --- & --- & --- & 2974 & 2638 & 2415 & 2265\\
2015-16 & --- & --- & --- & --- & --- & 3115 & 2726 & 2546\\
2018-19 & --- & --- & --- & --- & --- & --- & 3010 & 2708\\
2019-20 & --- & --- & --- & --- & --- & --- & --- & 2877\\
\end{tabular}
\end{table}

\label{r1:attrition}
Overall, we have an eight  year unbalanced panel with a total of 22,791 usable household-year observations.
The unconditional probability of a sample household observed in one wave not being observed in the next is about 0.14.  The lack of balance in the sample is driven less by households attriting, and more by the sample design, as some households were intentionally dropped and new ``refreshment'' samples were added in certain waves.  Most particularly, in the 2013--14 wave a third of the sample was replaced (this can be seen in \Tab{attrition}).   These sampling decisions are much more important to determining the probability of a random household remaining in the panel across rounds than the decisions or experience of the household itself, and in years where no part of the sample was intentionally dropped attrition is quite modest by the standards of panel surveys in low income countries (e.g., in 2019--20 attrition rates are less than 4\%).
\subsection{Data for Estimating MUEs}
\label{sec:org5846ad0}
\subsubsection*{Expenditure Data}
\label{sec:orgc4b7d6a}
Each wave of the LSMS includes a detailed elicitation of different kinds of food, plus two non-food items, ``cigarettes''  and ``other tobacco.'' If one takes the union of all items across all eight waves there are 175 distinct items; however, many of these simply involve differences in language (e.g., ``sweet bananas'' vs ``bogoya''), or what are actually differences in units  (e.g., ``matoke (heap)'' vs ``matoke (cluster)'').  Finally, some simply involve mis- or variant spelling (``mangos'' vs ``mangoes'').   Combining or reconciling these leads to a list of 130 distinct items.  Eliminating items which are not observed in every  round, or combining closely related items (``ground nuts (shelled)'' and ``ground nuts (pounded)'') gives us a final list of 41 goods, listed in \Fig{beta_estimates}.
\subsubsection*{Household Characteristics}
\label{sec:org7f92d6f}
\Tab{household_characteristics} reports on the characteristics of the sample households we use in our demand estimation, by year.  These characteristics are chosen because the household-level marginal utility of expenditures seems likely to depend not only on household size, but also on its composition.

\begin{table}[htbp]
\caption{\label{tab:household_characteristics}Mean characteristics of households in Uganda by year.  Figures in parentheses are standard deviations.  Households missing data on any of these characteristics or with inadequate data on food expenditures are excluded from the analysis.}
\centering
\begin{tabular}{rllllll}
\toprule
(\(t\)/\(N_t\)) & Girls & Boys & Women & Men & Rural & log HSize\\
\midrule
2005-06 & 1.62 & 1.61 & 1.22 & 1.06 & 0.73 & 1.53\\
2911 & (1.48) & (1.49) & (0.80) & (0.83) & (0.44) & (0.64)\\
2009-10 & 1.79 & 1.83 & 1.29 & 1.15 & 0.75 & 1.65\\
2760 & (1.53) & (1.60) & (0.81) & (0.91) & (0.43) & (0.60)\\
2010-11 & 1.91 & 1.91 & 1.34 & 1.18 & 0.78 & 1.70\\
2457 & (1.60) & (1.61) & (0.86) & (0.94) & (0.41) & (0.59)\\
2011-12 & 1.85 & 1.83 & 1.31 & 1.16 & 0.80 & 1.67\\
2687 & (1.57) & (1.58) & (0.80) & (0.91) & (0.40) & (0.58)\\
2013-14 & 1.62 & 1.62 & 1.27 & 1.13 & 0.74 & 1.57\\
2974 & (1.50) & (1.50) & (0.76) & (0.87) & (0.44) & (0.61)\\
2015-16 & 1.32 & 1.33 & 1.20 & 1.06 & 0.75 & 1.38\\
3115 & (1.39) & (1.39) & (0.76) & (0.89) & (0.43) & (0.71)\\
2018-19 & 1.51 & 1.46 & 1.26 & 1.13 & 0.75 & 1.53\\
3010 & (1.40) & (1.36) & (0.71) & (0.87) & (0.43) & (0.60)\\
2019-20 & 1.49 & 1.43 & 1.27 & 1.11 & 0.77 & 1.51\\
2877 & (1.40) & (1.32) & (0.70) & (0.86) & (0.42) & (0.60)\\
\midrule
Pooled & 1.63 & 1.62 & 1.27 & 1.12 & 0.76 & 1.56\\
22791 & (1.49) & (1.49) & (0.78) & (0.88) & (0.43) & (0.63)\\
\bottomrule
\end{tabular}
\end{table}
\section{Estimates of the CFE Expenditure System}
\label{sec:estimation}
\label{par:markets}
With the Ugandan data described above we estimate the \(\beta_j\) parameters and the values of \(\log\lambda_{it}\) for every household-year.  Some additional details determine our preferred specification.  First, we allow for demands to depend on the observed household characteristics \(z_{it}\) which appear in \Tab{household_characteristics}.  Second, we also need to take a stand on what constitutes a ``market,'' or the geographical area within which households are assumed to face the same prices.  Uganda is divided into four regions (central, east, west, north), and we assume that each region is a distinct market, thus allowing prices to vary by market-year.  To square this with the notation of the paper one should thus regard \(t\) as an index of market-years.  In our estimation of \Eq{estimating} this implies that \(t\) indexes \(4\times 8=32\) market-years, and that our specification thus includes \(32\times 41=1312\) market-year-good dummy variables.


\Fig{beta_estimates} shows our estimates of the \(\beta_j\) elasticities.  The ratios of the \(\beta\) coefficients are equal to ratios of income elasticities, implying that passion fruit is roughly three times as income elastic as cassava. Note that we can easily reject the hypothesis that these elasticities are all equal, as they would be in the special case of CRRA---there's a wide range of income elasticities across different goods, even for the same household, and the least elastic (those with low values of \(\beta_j\)) are what we might expect (starchy staples, salt).  Similarly, goods such as fresh milk, sweet bananas, coffee, oranges, and passion fruit all exhibit high elasticities.

\begin{figure}[htbp]
\centering
\includegraphics[width=.9\linewidth]{./Figures/beta_estimates.png}
\caption{\label{fig:beta_estimates}Estimates of Frischian elasticities \(\beta_j\).  These are proportional to income elasticities.  Horizontal bars are 95\% confidence intervals.}
\end{figure}

We estimate \(w=-\log\lambda_{it}\)  (we use the negative of the log MUE to make it easier to interpret the sign of the result---bigger is better) for every household in every year.
\Fig{w_by_year_joyplot} presents kernel density estimates of the distribution of  \(w\) for every year.   There is some evidence of this distribution shifting across years, though from these cross-sectional distributions we can't tell anything about how the position of a particular household changes over time.

\begin{figure}[htbp]
\centering
\includegraphics[width=.9\linewidth]{../Figures/Uganda/w_by_year_joyplot.png}
\caption{\label{fig:w_by_year_joyplot}Distribution of \(w\) by year.}
\end{figure}
\section{Direct Evidence on the Effects of Covariate Shocks}
\label{sec:effect_of_shocks}
To use estimated MUEs in a test of whether households are insured against covariate shocks, the only additional data required is data on those shocks, which we will use below to test whether covariate shocks affect differently situated households differently, or whether the effects of these shocks are captured entirely by changes in the prices faced by all households, consistent with the hypothesis of full insurance.

Yet this approach to measuring the effects of shocks  is indirect.  This is meant to be a virtue, because of the limited data requirements and simplicity of the test.  However, for the Ugandan panel we have additional data on (i) subjective reports on the effects of particular shocks; (ii) how households coped with those shocks; (iii) farm-gate prices for some locally produced goods; (iv) local market prices for a variety of foods; and (v) information on the quantities of food households consume.

We use these additional data to directly trace out the effects of covariate shocks in Appendix \ref{app:effect_of_shocks}.  Because of data limitations this account will be somewhat narrative and more informal than our main analysis.  Nevertheless, in the course of this narrative we present evidence that in Uganda:
\begin{description}
\item[{\S\ref{sec:org8b65133}}] Preferences are nonhomothetic, with expenditure shares that vary with budget and prices.  Thus changes in income or prices (whether induced by shocks or not) can be expected to induce changes in the composition of households' food expenditures.
\item[{\S\ref{sec:org16f84bd}}] Households report being affected by droughts, floods, agricultural pests, and adverse agricultural prices with some frequency. These shocks were indeed \emph{covariate}, in the sense that the probability of such reports covaries across space within a given year.
\item[{\S\ref{sec:orgbfce78a}}] Covariate shocks such as droughts, floods, and pests had a primary effect on agricultural production, which in turn led to reported declines in household income and consumption.  For example, 92\% of households who reported drought said it affected their production, 80\% said it affected their income, and 50\% said it affected their consumption.
\item[{\S\ref{app:shocks_and_farmgate_prices}}] Drought, pests, and adverse prices (but not floods) tended to lead to statistically significant increases in local farm-gate prices.
\item[{\S\ref{sec:org5a8a1d1}}] Households which experienced \emph{covariate} shocks responded primarily through methods of self-insurance: reducing consumption, drawing on savings, and increasing labor supply, all strategies one might employ in the face of increased prices.  In contrast, those who experienced \emph{idiosyncratic} shocks most often relied on help from friends and family.
\item[{\S\ref{app:shocks_and_relative_prices}}] Most relative \emph{market} prices for different foods were significantly different in places that experienced covariate shocks (including floods) versus those that did not.
\item[{\S\ref{sec:food_quantities}}] Recent experience of drought, floods, or pests had significant effects on the composition of diet, with negative consequences for dietary diversity and for nutrition.  In particular foods from animal sources (meat, dairy, eggs, certain cooking oils) all tend to decline.  As animal products are the only important sources of vitamin B-12, this nutrient can also be presumed to decline.
\end{description}
\section{Tests of Risk Sharing in the Face of Covariate Shocks}
\label{sec:tests}
In this section we want to actually conduct the risk-sharing tests described above.  We'll use two different approaches.  The first is the classic test assuming CRRA utility: for these the dependent variable is the log of total expenditures.  The second is the test assuming our more general CFE preferences; for these the dependent variable is \(w=-\log\lambda\).
\subsection{Effects of Shocks on Welfare Last Year}
\label{sec:org9675c9b}


\begin{table}
\begin{threeparttable}
\caption{Effects of different shocks on welfare measures $w$ and $\log x$.\tnote{$N$,$T$,$H$,$X$,$FE$}}
\label{tab:effects_of_different_shocks_last_year}
{\small
\begin{tabular}{lD{.}{.}{-1}@{}D{.}{.}{-1}@{}}
\toprule
Shock & \hd{$w$ (CFE)} & \hd{\(\log x\) (CRRA)}\\
\midrule
Health & -0.059\tnote{**} & -0.028\\
 & (0.024) & (0.018)\\
Theft & 0.021 & 0.020\\
 & (0.033) & (0.026)\\
Death (non-earner) & 0.021 & 0.019\\
 & (0.050) & (0.035)\\
Death of earner & 0.031 & 0.009\\
 & (0.089) & (0.065)\\
Income & 0.050\tnote{***} & 0.071\tnote{***}\\
 & (0.007) & (0.006)\\
\midrule
Drought & 0.010 & 0.046\tnote{***}\\
 & (0.015) & (0.011)\\
Floods & 0.035 & 0.097\tnote{***}\\
 & (0.034) & (0.027)\\
Pests & 0.041 & 0.095\tnote{***}\\
 & (0.034) & (0.024)\\
Prices & -0.043 & 0.103\tnote{***}\\
 & (0.038) & (0.028)\\
\bottomrule
\end{tabular}}
\begin{tablenotes}
\item[*]{\(p\)<0.10;} \item[**]{\(p\)<0.05.} \item[***]{\(p\)<0.01.}
\item[$N$]{Total observations are 22791 unless otherwise noted.}
\item[$T$]{All regressions involve 8 rounds and 4 markets.}
\item[$H$]{A total of 5604 distinct households, unless otherwise noted (but not all households appear in all rounds).}
\item[$X$]{All regressions include the number of boys, girls, men, women, log of household size, and a ``rural'' dummy as additional covariates.}
\item[FE]{Household Fixed Effects: Yes; Market-year Effects: Yes}
\item[I]{The regression on log income has $N=9721$, and $H=3851$.}
\end{tablenotes}
\end{threeparttable}
\end{table}

\label{r2:fix_d}
The risk-sharing test we're concerned with is meant to provide a test of the null hypothesis that \(\E(\log\lambda|p,z)=\E(\log\lambda|p,z,\mbox{Shocks})\); that is, whether shocks affect log MUEs after we control for possible changes to prices or household characteristics.  If demands are consistent with the CFE specification, then the welfare measures \(w\) we've constructed are estimates of \(-\log\lambda\), and we are justified in using \(w\) as the dependent variable in the two-way fixed effects regressions we've described.   When \(\beta_j=\bar\beta\) for all goods \(j\), then CFE demands will coincide with the special case of CRRA, and we will have \(-\log\lambda\) proportional to \(\log x\).

\Tab{effects_of_different_shocks_last_year} describes the effects of different shocks on both \(w\) (equation \Eq{risksharing}, the CFE risk-sharing regression) and on \(\log x\) (equation \Eq{crra_risksharing}, the CRRA risk-sharing regression).  Estimated \(w\)s are scaled so as to have the same variance as observed \(\log x\); this makes it possible to compare the magnitude of estimated coefficients across the two specifications.  Reported standard errors here and elsewhere are robust \autocite{arellano87}.

Recall that if preferences are \emph{not} actually CRRA then we'd expect the disturbance term in the mis-specified TWFE CRRA regression to depend on prices.  In particular, if income and price elasticities are not equal across all goods then we would expect that a regression of log total food expenditures on shocks which increased food expenditures would yield positive estimated values of \(\delta\).   Similarly, if Frischian elasticities vary across goods but are constant as in the CFE system then we would expect the time-market effects in the regression to correctly capture the effects of prices on expenditures, so that estimated values of \(\delta\) should be zero (if there's full risk-sharing).

The results of \Tab{effects_of_different_shocks_last_year} confirm this reasoning rather dramatically.   In fact, \emph{every} covariate shock has a significant positive coefficient in the CRRA regressions, while \emph{no} covariate shock has a significant coefficient in the CFE regressions, just as we would predict if there's full insurance and demands are CFE.  This is strong evidence in favor of the view that these negative covariate shocks increase local consumption expenditures by causing an increase in some local prices,\footnote{See \App{shocks_and_farmgate_prices} for direct evidence on this point.} and that inclusion of time-market effects in the CRRA specification does not control for these price effects correctly.

\label{par:income}
Among the idiosyncratic shocks, income\footnote{Construction of the income variable is described in Online Appendix \ref{ref:income}.\label{fn:income}} is significant, has the expected sign, and has a magnitude similar to that found in other Townsend-style tests of full insurance.  Thus, our more general specification of MUEs does not alter the usual conclusion that households are not fully insured against idiosyncratic variation in income. No other coefficients in either specification are significant, with the sole exception of Health for CFE.  This last should not be a surprise, and neither should it be interpreted as a rejection of the risk-sharing model, as our estimation of the CFE demands did not include any information on health as a household characteristic.  If health affects demands, in our specification it can do so either via a shock to the budget and thus \(w\) (which would be at odds with full insurance, but not the CFE demand specification) or via the disturbance term in the demand equations, which explicitly depends on unobserved household characteristics such as health.
\subsection{Effects of Covariate Shocks on Welfare by Month}
\label{sec:orgb97699b}

The data we have on expenditures is for the past week; the data we have on shocks is for the past year.  Could this somehow cause the apparently abberant effects of  negative shocks on expenditures in the CRRA regressions, rather than something to do with prices?  Perhaps spending increases as people attempt recovery from the shock?\footnote{Though here it's worth remembering that we measure non-durable expenditures, mostly food, as distinct from investment.\label{fn:recovery}}  We examine this by considering shocks at different temporal proximities to the expenditures.  We construct a dummy variable which takes the value one if there's a reported covariate shock within the \(m\) months prior to the interview, and then re-estimate our risk-sharing regressions allowing \(m\) to vary from zero months up to twelve months.

\begin{figure}[htbp]
\centering
\includegraphics[width=.9\linewidth]{./Figures/covariate_shocks_by_month.png}
\caption{\label{fig:covariate_shocks_by_month}Effects of any covariate shock within the last \(m\) months on welfare measures, from a two-way panel regression otherwise specified as in \Tab{effects_of_different_shocks_last_year}.   Scale is in standard deviations of the dependent variable (\(w\) or \(\log x\)).  Bars indicate 95\% confidence intervals.}
\end{figure}

Results are reported in Figure \ref{fig:covariate_shocks_by_month}. This figure provides even stronger evidence favoring the CFE over the CRRA risk-sharing specifications.  For the CRRA (\(\log x\)) specification, having had any covariate shock 3--12 months previous has a significant positive effect on log consumption, in every month.  In contrast, for the CFE (\(w\)) specification, having any covariate shock does not have a significant effect on \(w\), regardless of the period of time.   The full risk-sharing hypothesis holds that covariate shocks should influence \(w\) only through prices, consistent with the evidence for the \(w\)-based regressions in the table.


\begin{table}
\begin{threeparttable}
\caption{Effects of different shocks within the last $m$ months on current log consumption expenditures.\tnote{$N$,$T$,$H$,$X$,FE}}
\label{tab:shocks_on_logx}
{\scriptsize
\begin{tabular}{rD{.}{.}{-1}@{}D{.}{.}{-1}@{}D{.}{.}{-1}@{}D{.}{.}{-1}@{}|D{.}{.}{-1}@{}D{.}{.}{-1}@{}D{.}{.}{-1}@{}D{.}{.}{-1}}
\toprule
 &  &  & \hd{Death} & \hd{Death} &  &  &  & \\
Months & \hd{Health} & \hd{Theft} & \hd{(non-earner)} & \hd{(earner)} & \hd{Drought} & \hd{Floods} & \hd{Pests} & \hd{Prices}\\
\midrule
0 & 0.01 & 0.05 & 0.09 & 0.77\tnote{***} & -0.07 & 0.28\tnote{***} & 0.30\tnote{**} & -0.21\\
 & (0.08) & (0.09) & (0.20) & (0.22) & (0.10) & (0.10) & (0.12) & (0.20)\\
1 & -0.01 & 0.01 & -0.02 & 0.22 & -0.01 & 0.10 & 0.20\tnote{***} & 0.07\\
 & (0.04) & (0.05) & (0.10) & (0.14) & (0.05) & (0.09) & (0.07) & (0.10)\\
2 & -0.03 & -0.01 & 0.02 & 0.35\tnote{**} & -0.02 & 0.12\tnote{*} & 0.18\tnote{***} & 0.09\\
 & (0.03) & (0.04) & (0.08) & (0.18) & (0.03) & (0.06) & (0.06) & (0.06)\\
3 & -0.02 & -0.01 & 0.04 & 0.29\tnote{**} & 0.00 & 0.18\tnote{***} & 0.16\tnote{***} & 0.10\tnote{**}\\
 & (0.03) & (0.04) & (0.06) & (0.14) & (0.02) & (0.05) & (0.05) & (0.05)\\
4 & -0.03 & -0.02 & 0.04 & 0.10 & 0.02 & 0.14\tnote{***} & 0.14\tnote{***} & 0.12\tnote{***}\\
 & (0.03) & (0.03) & (0.05) & (0.12) & (0.02) & (0.04) & (0.04) & (0.05)\\
5 & -0.03 & -0.02 & 0.05 & 0.05 & 0.05\tnote{***} & 0.12\tnote{***} & 0.16\tnote{***} & 0.10\tnote{**}\\
 & (0.02) & (0.03) & (0.05) & (0.11) & (0.02) & (0.04) & (0.04) & (0.04)\\
6 & -0.04\tnote{*} & -0.01 & 0.03 & 0.04 & 0.06\tnote{***} & 0.09\tnote{**} & 0.15\tnote{***} & 0.12\tnote{***}\\
 & (0.02) & (0.03) & (0.04) & (0.09) & (0.01) & (0.03) & (0.03) & (0.04)\\
7 & -0.02 & 0.01 & 0.00 & 0.03 & 0.05\tnote{***} & 0.07\tnote{**} & 0.13\tnote{***} & 0.11\tnote{***}\\
 & (0.02) & (0.03) & (0.04) & (0.09) & (0.01) & (0.03) & (0.03) & (0.03)\\
8 & -0.02 & -0.00 & 0.02 & 0.05 & 0.05\tnote{***} & 0.07\tnote{**} & 0.13\tnote{***} & 0.11\tnote{***}\\
 & (0.02) & (0.03) & (0.04) & (0.08) & (0.01) & (0.03) & (0.03) & (0.03)\\
9 & -0.02 & 0.00 & 0.02 & 0.03 & 0.06\tnote{***} & 0.07\tnote{**} & 0.12\tnote{***} & 0.10\tnote{***}\\
 & (0.02) & (0.03) & (0.04) & (0.07) & (0.01) & (0.03) & (0.03) & (0.03)\\
10 & -0.02 & 0.01 & 0.03 & 0.02 & 0.05\tnote{***} & 0.08\tnote{***} & 0.12\tnote{***} & 0.11\tnote{***}\\
 & (0.02) & (0.03) & (0.04) & (0.07) & (0.01) & (0.03) & (0.03) & (0.03)\\
11 & -0.03 & 0.01 & 0.03 & 0.03 & 0.05\tnote{***} & 0.08\tnote{***} & 0.11\tnote{***} & 0.10\tnote{***}\\
 & (0.02) & (0.03) & (0.04) & (0.07) & (0.01) & (0.03) & (0.02) & (0.03)\\
12 & -0.03 & 0.02 & 0.02 & 0.01 & 0.05\tnote{***} & 0.10\tnote{***} & 0.09\tnote{***} & 0.10\tnote{***}\\
 & (0.02) & (0.03) & (0.04) & (0.06) & (0.01) & (0.03) & (0.02) & (0.03)\\
\bottomrule
\end{tabular}}
\begin{tablenotes}
\item[*]{\(p\)<0.10;} \item[**]{\(p\)<0.05.} \item[***]{\(p\)<0.01.}
\item[$N$]{Total observations are 22791.}
\item[$T$]{All regressions involve 8 rounds and 4 markets.}
\item[$H$]{A total of 5604 distinct households (but not all households appear in all rounds).}
\item[$X$]{All regressions include the number of boys, girls, men, women, log of household size, and a ``rural'' dummy as additional covariates.}
\item[FE]{Household Fixed Effects: Yes; Market-year Effects: Yes}
\end{tablenotes}
\end{threeparttable}
\end{table}

\begin{table}
\begin{threeparttable}
\caption{Effects of different shocks within the last $m$ months on current $w=-\log\lambda$.\tnote{$N$,$T$,$H$,$X$,FE}}
\label{tab:shocks_on_w}
{\scriptsize
\begin{tabular}{rD{.}{.}{-1}@{}D{.}{.}{-1}@{}D{.}{.}{-1}@{}D{.}{.}{-1}@{}|D{.}{.}{-1}@{}D{.}{.}{-1}@{}D{.}{.}{-1}@{}D{.}{.}{-1}}
\toprule
 &  &  & \hd{Death} & \hd{Death} &  &  &  & \\
Months & \hd{Health} & \hd{Theft} & \hd{(non-earner)} & \hd{(earner)} & \hd{Drought} & \hd{Floods} & \hd{Pests} & \hd{Prices}\\
\midrule
0 & -0.05 & 0.24\tnote{**} & -0.07 & 1.02\tnote{**} & -0.38\tnote{**} & 0.36\tnote{*} & -0.09 & -0.52\tnote{***}\\
 & (0.09) & (0.12) & (0.31) & (0.45) & (0.15) & (0.21) & (0.15) & (0.18)\\
1 & 0.01 & 0.05 & -0.08 & 0.41\tnote{**} & -0.06 & 0.10 & 0.12 & 0.06\\
 & (0.05) & (0.06) & (0.15) & (0.20) & (0.06) & (0.10) & (0.10) & (0.15)\\
2 & 0.02 & 0.06 & -0.09 & 0.49\tnote{**} & 0.00 & 0.06 & 0.11 & 0.06\\
 & (0.04) & (0.05) & (0.11) & (0.21) & (0.04) & (0.07) & (0.08) & (0.12)\\
3 & 0.01 & 0.01 & 0.03 & 0.37\tnote{**} & -0.00 & 0.10\tnote{*} & 0.11\tnote{*} & -0.09\\
 & (0.04) & (0.05) & (0.09) & (0.18) & (0.03) & (0.06) & (0.06) & (0.09)\\
4 & -0.03 & -0.01 & 0.02 & 0.11 & 0.01 & 0.12\tnote{**} & 0.08 & -0.06\\
 & (0.03) & (0.04) & (0.08) & (0.16) & (0.03) & (0.05) & (0.06) & (0.07)\\
5 & -0.04 & -0.01 & 0.02 & 0.08 & 0.02 & 0.08\tnote{*} & 0.10\tnote{**} & -0.10\tnote{*}\\
 & (0.03) & (0.04) & (0.07) & (0.14) & (0.02) & (0.04) & (0.05) & (0.06)\\
6 & -0.04 & -0.01 & 0.03 & 0.07 & 0.02 & 0.04 & 0.09\tnote{*} & -0.08\\
 & (0.03) & (0.04) & (0.06) & (0.12) & (0.02) & (0.04) & (0.05) & (0.05)\\
7 & -0.04 & 0.01 & 0.01 & 0.06 & 0.00 & 0.02 & 0.05 & -0.07\\
 & (0.03) & (0.04) & (0.06) & (0.12) & (0.02) & (0.04) & (0.04) & (0.05)\\
8 & -0.04 & 0.00 & 0.03 & 0.10 & 0.01 & 0.01 & 0.07 & -0.07\\
 & (0.03) & (0.04) & (0.06) & (0.11) & (0.02) & (0.04) & (0.04) & (0.04)\\
9 & -0.04 & 0.00 & 0.07 & 0.08 & 0.01 & 0.03 & 0.07\tnote{*} & -0.07\tnote{*}\\
 & (0.03) & (0.03) & (0.06) & (0.10) & (0.02) & (0.04) & (0.04) & (0.04)\\
10 & -0.04 & 0.02 & 0.05 & 0.05 & 0.01 & 0.03 & 0.07\tnote{*} & -0.05\\
 & (0.03) & (0.03) & (0.05) & (0.10) & (0.02) & (0.03) & (0.04) & (0.04)\\
11 & -0.05\tnote{**} & 0.02 & 0.05 & 0.08 & 0.01 & 0.03 & 0.05 & -0.04\\
 & (0.03) & (0.03) & (0.05) & (0.09) & (0.01) & (0.03) & (0.04) & (0.04)\\
12 & -0.06\tnote{**} & 0.02 & 0.02 & 0.03 & 0.01 & 0.04 & 0.04 & -0.04\\
 & (0.02) & (0.03) & (0.05) & (0.09) & (0.01) & (0.03) & (0.03) & (0.04)\\
\bottomrule
\end{tabular}}
\begin{tablenotes}
\item[*]{\(p\)<0.10;} \item[**]{\(p\)<0.05.} \item[***]{\(p\)<0.01.}
\item[$N$]{Total observations are 22791.}
\item[$T$]{All regressions involve 8 rounds and 4 markets.}
\item[$H$]{A total of 5604 distinct households (but not all households appear in all rounds).}
\item[$X$]{All regressions include the number of boys, girls, men, women, log of household size, and a ``rural'' dummy as additional covariates.}
\item[FE]{Household Fixed Effects: Yes; Market-year Effects: Yes}
\end{tablenotes}
\end{threeparttable}
\end{table}

In \Tab{effects_of_different_shocks_last_year} we tried to measure the effects of any covariate shock during the last year on \(\log x\) and \(w\), and found positive effects on the former and no significant on the latter.  Then in \Tab{shocks_on_logx} and \Tab{shocks_on_w} we tried to see if we took into account how recent the shock was.  This didn't affect the conclusion that \(\log x\) increases in response to negative covariate shocks, while \(w\) doesn't.   We next disaggregate by both the recency and type of the shock---perhaps some kinds of covariate shocks are well insured, but others are not, or the timing of impact differs by shock.  \Tab{shocks_on_logx} reports the same kind of regressions described by Figure \ref{fig:covariate_shocks_by_month} which allow for shock windows of zero through twelve months, but keeps the different kinds of shocks separate, as in \Tab{effects_of_different_shocks_last_year} (except that income is dropped, as we don't have detailed data on the timing of its receipt).   As in that table, most (42 of 52) of the covariate shock coefficients (the last four columns) are significant and positive in the CRRA specification at a 95\% percent level of confidence, while only four (of 52) are positive and significant at this level for the CFE specification reported in the subsequent \Tab{shocks_on_w}.  This is slightly higher than the number of false positives (2.6) we'd expect under the null, but constitutes very weak evidence against the full insurance hypothesis.
\subsection{Effects of Covariate Shocks on Average Welfare}
\label{sec:org7ce1c72}

Using a two-way fixed effects regression we've seen that covariate shocks have at most a small effect on \(w\), save for a shock in the same month. We've argued that this is because the main mechanism by which covariate shocks affect MUEs is via prices, and the two-way fixed effects regression controls for these if demands are CFE.    What if we \emph{don't} control for prices?  A panel regression incorporating only household fixed effects would then reveal the effects that shocks have on MUEs via prices.    Of course, attribution becomes problematical, since the incidence of covariate shocks across years and markets will be correlated not only with prices, but possibly other unobserved shocks.
Accordingly, Figure \ref{fig:covariate_shocks_by_month_one_way} shows the results from a series of regressions of \(w\) and \(\log x\) on the number of reported covariate shocks within the last \(m\) months, just as in Figure \ref{fig:covariate_shocks_by_month}, but in this case only as a ``one-way'' panel estimator, controlling for household fixed effects but \emph{not} time-market effects.

\begin{figure}[htbp]
\centering
\includegraphics[width=.9\linewidth]{../Figures/Uganda/covariate_shocks_by_month_one_way.png}
\caption{\label{fig:covariate_shocks_by_month_one_way}Effects of any covariate shock within the last \(m\) months on welfare measures with household fixed effects, but otherwise specified as in \Tab{effects_of_different_shocks_last_year}.   Scale is in standard deviations of the dependent variable (\(w\) or \(\log x\)).  Bars indicate 95\% confidence intervals.}
\end{figure}

Here \(w\) falls with reported ``negative'' covariate shocks (drought, floods, pests, prices) in the way we would expect, with estimated coefficients significantly negative for some months at more distant intervals (anything from seven to twelve months).  In contrast, ``negative'' covariate shocks have a uniformly \emph{positive} effect on expenditures, significant at all but the shortest intervals.  This is once again consistent with the view that these shocks increase local prices of goods with inelastic demands, leading to an increase in expenditures but a decrease in consumption of those goods.
\section{Conclusion}
\label{sec:conclusion}
\subsection{Implications for Tests of Full Risk Sharing}
\label{sec:implications}
Using standard risk-sharing regressions to test for insurance against covariate shocks (shocks that affect prices) can yield very surprising results---in the example of Uganda, droughts, floods, pests, and adverse changes in prices appear to be at least partially uninsured, but to \emph{improve} welfare as measured by ``real'' consumption expenditures.

We argue that these surprising results are a consequence of the standard risk-sharing regressions actually being a \emph{joint} test of full insurance along with preferences being homothetic---in this case changes in relative prices affect welfare only via a single scalar price index.  There is extremely strong evidence against utility being homothetic, starting with \textcite{engel1857}, since homothetic utility implies that all demands must have an income elasticity of one.   This doesn't matter much when one tests risk-sharing with respect to idiosyncratic shocks, since almost by definition these won't affect prices.  But it can matter very much when prices change, since increases in prices can very easily increase total ``real'' expenditures while utility actually falls.

There are two great virtues of the risk-sharing regression framework.  The first is its theoretical simplicity---it's really  a model of households' marginal utilities of expenditure (MUEs), and in a world with full insurance these have a simple factor structure which is easily tested using panel methods with two-way fixed effects.  The second is that assuming homothetic utility allows us to write the MUE as a simple function of nothing but total expenditures (typically \(x^{-\gamma}\)).  And there is plenty of carefully collected data on household expenditures which one can use to construct these MUEs for dozens of countries over many years.

However, if we want to understand insurance against covariate shocks or price changes, the evidence that we need to abandon homothetic utility is overwhelming.  So are there other ways to construct estimates of MUE that depend only on expenditures?  We show that there are.  If we use data on \emph{item}-level expenditures, instead of adding these up to obtain a total, there's information in the composition of these expenditures which can be used to estimate the MUE \(\lambda\).  A condition for being able to use expenditure data to estimate \(\lambda\) is that the expenditure system must be separable in \(\lambda\) and prices.  When we impose this separability and exploit the homogeneity of expenditures in prices, we show that the expenditure system can be written in a form called a generalized Pexider functional equation.

We exploit results from the theory of functional equations to obtain the entire class of possible demand systems consistent with inferring the MUE from nothing more than expenditures.  These solutions fall into two families of semiparametric demands---one corresponds to a generalization of CRRA utility we call ``Constant Frisch Elasticity'' (CFE), while the other corresponds to a generalization of Stone-Geary utility.  Both admit nonhomothetic preferences and very flexible responses to changes in relative prices, but only the CFE system is easily estimated.

We use an eight round panel dataset from Uganda to estimate MUEs from expenditure data.  We also obtain estimates of elasticities which emphatically reject the hypothesis of unitary income elasticities (a feature of CRRA demands).  Using self-reported data on both covariate and idiosyncratic shocks, we estimate the risk-sharing regressions, using as dependent variables (a) the logarithm of total expenditures, as in the usual CRRA case; and (b) the values of \(w=-\log\lambda\) estimated from the CFE expenditure system in a two-way panel regression (which also includes household demographics).   As theory predicts, in this specification covariate shocks and adverse price changes have significant  effects on log total expenditures (even if perfectly observed)\footnote{If in addition observed expenditures disproportionally represent goods with low income elasticities (such as food, per Engel's Law), then covariate shocks will tend to have a positive effect on total expenditures, as observed here.\label{fn:engel}}, and \emph{no} significant effect on the CFE estimates of MUE, as these are constructed to account for any changes in prices.
\subsection{Insurance Against Covariate Shocks in Uganda}
\label{sec:insurance_in_uganda}
We think of household expenditures depending on three things: prices, budget, and household characteristics.   Idiosyncratic shocks may affect expenditures by changing the household's budget (e.g., via a shortfall in income), or perhaps by changing household characteristics (e.g., someone dies).   But a shock to income need not affect the household's budget, as risk associated with idiosyncratic shocks can be \emph{shared} with others.  Covariate shocks may also operate via the channels of budget or household characteristics, but in addition may affect the common prices households face.   Changes to common prices are in effect already shared---if the effect of a shock such a drought is experienced solely through changes in prices then there's no scope for sharing to improve efficiency.  But if covariate shocks affect differently situated households differently, then it will be efficient for the these idiosyncratic impacts to be shared.

We've offered direct evidence (\Sec{effect_of_shocks} and \App{effect_of_shocks}) that in Uganda shocks such as drought, floods, and pests have significant effects on prices, and change households' diets so as to reduce dietary diversity, with negative consequences for nutrition.  That is to say, these shocks affect welfare and consumption in ways that are measurably bad.   But are these bad effects of shocks transmitted idiosyncratically via channels of budget or household characteristics, or are the bad effects shared via changes in prices?

The MUEs we've constructed for Uganda allow for the flexible control of changes in common prices; this is in contrast to the CRRA approach, which only permits prices to influence expenditures via a single scalar price index.  Our innovation allows us test whether the idiosyncratic impacts of covariate shocks are insured.  While we find that covariate shocks affect MUEs, the principal channel seems to be via changes to common prices; there's very little evidence that covariate shocks such as droughts, floods, or pests cause idiosyncratic variation in MUEs.

These results have implications for policy.  For example, they suggest that programs meant to provide relief from these kinds of covariate shocks should focus on providing that relief to communities, and not try to target particular households.  And because we find that the impacts of covariate shocks are broadly shared by the people within a given market-region, who face common prices, there are benefits to trying to make those markets as broad as possible.  Better integrating markets across regions of Uganda or better integrating Uganda's markets into the broader African or world economy would reduce the impact that local covariate shocks have on prices.
\subsection{Coda: MUEs are Everywhere}
\label{sec:mues_everywhere}
This paper has focused on the case of full risk-sharing.  This application may seem rather narrow, and so of limited interest.  However, the construction of estimates of MUEs is independent of the efficient risk-sharing hypothesis, so estimated MUEs could be used to estimate and test any of a large variety of dynamic life-cycle models, following a program suggested by \textcite{blundell98}.  The usual consumption Euler equation is, after all, a statement about MUEs across time, and ratios of MUEs across periods give us a way to calculate intertemporal marginal rates of substitution, free of the usual homotheticity assumptions.

The practice of using CRRA preferences and measuring MUEs as an expenditure aggregate raised to some negative power is extremely common across many fields of economics.  The novel tool developed in this paper is an improved way to measure MUEs.  We've constrained ourselves to using nothing more than the same expenditure data necessary to construct the expenditure aggregates typically employed by those working with CRRA or other homothetic preferences.   The central idea is that CFE MUEs can be easily computed from the same data used to construct CRRA MUEs, and so serve as a more general drop-in replacement.

Consider first applications in other fields that use Townsend-style tests of risk-sharing.  Though testing for risk-sharing across households is the classic approach, a similar setup is used in other settings, for example to test for sharing across people or generations in the unitary household model \citep[e.g.,][]{altonji-etal92,hayashi-etal96,dercon-krishnan00a,lise-yamada19,theloudis-etal25}.\footnote{MUEs also play a central role in the literature on the collective household \autocite[e.g.,][]{browning-etal94}.  This literature also suggests a way in which one could extend preferences to incorporate other-regarding motives or public goods.\label{fn:collective}}   The technique we attribute to Townsend arguably started in macro or international \autocite{mace91,cochrane91,obstfeld93,attanasio-davis96} and is still used \citep[e.g.,][]{degiorgi-etal20,lee-lee25}.

Then there's the development of a literature on ``partial insurance'' \autocite{blundell-etal08}, which models dynamics in a permanent income-life cycle  world where insurance is imperfect, but households have access to (at least) some form of credit market or intertemporal storage.   The derivation of the key estimating equation is somewhat involved, but delivers an estimating equation which looks just like a differenced version of the Townsend regression with changes in log consumption as a dependent variable, and measures of each of permanent and transitory income as ``shocks.''  This has many  of the same features as the Townsend risk-sharing test---a possibly mis-measured MUE, with mis-measurement which can contaminate the interpretation of the time dummy as controlling for changes in prices (returns).   Most examples of the use of this setup come from the  macro literature \citep[e.g.,][]{kaplan-violante10,arellano-etal17}, but there are also applications in development \citep{bazzi-etal15,santaeulalia-zheng18}.

Finally, there's scope for use of MUEs from the CFE demand system even in applications where the author may  not be explicitly thinking in terms of MUEs.   For example, there are many applications that start with a Rubin causal model with the intention of estimating treatment effects on (log) consumption.   But this often leads to a TWFE estimator that  also has a structural interpretation.  This is particularly the case in a large number of RCTs in development, where it's very common to use a double-difference (a special case of TWFE) to estimate a treatment effect on total consumption expenditures (in levels or in logs).  In these applications what we're calling ``risk-sharing'' might be interpreted as ``spillovers,'' and the use of fixed effects might be interpreted as a test of ``parallel trends,'' but it's essentially the same econometric specification that Townsend employed.  In this case there will also be a MUE \& risk-sharing interpretation, with the benefit that if assignment is random there's also a causal interpretation \citep[e.g.,][]{karlan-etal14,banerjee-etal15microfinance,haushofer-shapiro16,angelucci-etal18}.





\clearpage
\bibliography{covariate_risksharing}
\vfill\eject
\appendix
\renewcommand\thetable{\thesection.\arabic{table}}
\renewcommand\thefigure{\thesection.\arabic{figure}}
\renewcommand\thelemma{\thesection.\arabic{lemma}}
\renewcommand\thetheorem{\thesection.\arabic{theorem}}
\renewcommand\thecorollary{\thesection.\arabic{corollary}}
\setcounter{table}{0}    % Reset table numbers for appendix
\setcounter{figure}{0}    % Reset table numbers for appendix
\setcounter{theorem}{0}    % Reset table numbers for appendix
\section{Proof of Theorem 1}
\label{sec:org4896922}
In this appendix we provide a proof of Theorem 1.  We first supply a lemma pertaining to the homogeneity of \(\lambda\)-separable expenditure systems, and then provide solutions to the generalized Pexider equation.  With these preliminary results in hand we establish that any rationalizable system of expenditures which is \(\lambda\)-separable takes the form of the generalized Pexider equation, and map the general solutions of the equation into corresponding demands and utilities.
\subsection{A Lemma Pertaining to Homogeneity}
\label{sec:orgb971a1a}

\begin{lemma}
If demands are rationalizable (\Defn{valid}) and expenditures on good \(j\) are \(\lambda\)-separable (\Defn{xlseparable}), then
  the functions \(\phi_j\), \(a_j\) and \(b_j\) are either all logarithmic
  or \(\phi_j\) and \(a_j\) are both positive homogeneous of some degree
  \(\beta_j\), while \(b_j\) is positive homogeneous of degree \(-\beta_j\).
\label{lem:xhomogeneous}
\end{lemma}

\begin{proof}
Rationalizable expenditures \(x_j\) must be homogeneous of degree one
in \((p,1/\lambda)\).  The definition of \(\lambda\)-separability implies that
  \[
     x_j(p,\lambda) = \phi_j^{-1}\left(a_j(p) + b_j(\lambda)\right)
  \]
  is similarly homogeneous of degree one.  The function \(\phi_j\) must then
either be homogeneous of degree \(\beta_j\), with
\(\phi_j(x_j)=x_j^{\beta_j}\), or else \(\phi_j(x_j)=\log(x_j)\).  In either case
Frisch quantities can be written as
  \[
     c_j(p,\lambda)=f_j(p\lambda)=\frac{1}{p_j}\phi_j^{-1}\left(a_j(p) + b_j(\lambda)\right) - d_j(p)
  \]
  for some function \(d_j\) homogeneous of degree zero.

We consider the power and logarithmic cases in turn.

First suppose that \(\phi_j(x)=x^{\beta_j}\).  Then the sum \(a_j+b_j\)
must also be homogeneous of degree \(\beta_j\) in \((p,1/\lambda)\), and
the individual functions \(a_j\) and \(b_j\) respectively either
homogeneous of degree \(\beta_j\) and \(-\beta_j\) or else the zero
function.  It follows that
\(f_j(p,\lambda)=\phi_j^{-1}\left(a_j(p)/p_j^{\beta_j} +
  b_j(\lambda)/p_j^{\beta_j}\right) - d_j(p)\), and that
\(a_j(p)/p_j^{\beta_j}\) and \(b_j(\lambda)/p_j^{\beta_j}\) are either
zero or positive homogeneous of degree zero, so that
  \[
     \left( a_j(p\theta) + b_j(\lambda/\theta)\right) = \theta^{\beta_j}(a_j(p) + b_j(\lambda)) = \theta^{\beta_j}a_j(p) + \theta^{\beta_j}b_j(\lambda)
  \]
for any positive scalar \(\theta\).
Differentiating this with respect to \(1/\lambda\) establishes that \(b'_j\) is
homogeneous of degree \(\beta_j-1\), so that \(b_j\) is homogeneous of
degree \(\beta_j\) (by Euler's theorem of positive homogeneous
functions).  A similar argument involving the gradient with respect
to \(p\) establishes the same for \(a_j\).

For the logarithmic case, \(\phi(x_j)=\log(x_j)=\log(p_j)+\log(c_j)\) implies that
  \[
    f_j(p,\lambda) + d_j(p) = \exp\left(a_j(p) + b_j(\lambda) -\log(p_j) \right)
  \]
  which must be positive homogeneous of degree zero in \((p,1/\lambda)\).  This implies
that for any \(\theta>0\)
  \[
     a_j(\theta p) + b_j(\lambda/\theta ) - \log(\theta p_j) = a_j(p) + b_j(\lambda) - \log(p_j),
  \]
  which in turn implies that
  \[
     a_j(\theta p) + b_j(\lambda/\theta)  = a_j(p) + b_j(\lambda) + \log(\theta),
  \]
  implying that both \(a_j\) and \(b_j\) are linear in logs of (p,\(\lambda\)).
\end{proof}
\subsection{Generalized Pexider Equation Applied to Vector Spaces}
\label{sec:orgb99fa20}
We now introduce our main tool for solving the functional equations
implied by separability and rationalizability; this tool is an application of
what is called the generalized Pexider equation, when the domain of
application is limited to real vector spaces.

Consider the generalized Pexider equation
  \begin{equation}
\label{eq:generalized_pexider}
k(x+y) = g(x)l(y) + h(y)
\end{equation}
where
  \begin{align}
g(x)=\frac{k(x)-h(0)}{l(0)} \\
\varphi(y)=\frac{l(y)}{l(0)} \label{eq:varphi}\\
\psi(y)=h(y)-h(0)\frac{l(y)}{l(0)} \label{eq:psi}\\
k(x+y)=k(x)\varphi(y) + \psi(y) \label{eq:psi2}\\
\kappa(x)=k(x) - k(0)           \label{eq:kappa}\\
\kappa(x+y)=\kappa(x)\varphi(y) + \kappa(y).  \label{eq:kappa2}
\end{align}

  Next we give statements of two related lemmata.  The first is just a
statement of the solution of the well-known functional equation of
Cauchy applied to real vector spaces; the second is a statement of
the solution to what is sometimes called Cauchy's exponential
equation, again for real vector spaces.
\begin{lemma}
Let \(f:\R^n\rightarrow\R^m\), with \(f\) continuous at a point.  Then
if
  \begin{equation}
\label{eq:cauchyeqn}
   f(x+y)=f(x)+f(y)
\end{equation}
then \(f(x)=Cx\) for some constant \(m\times n\) matrix \(C\).
\label{lem:cauchyeqn}
\end{lemma}
Also
\begin{lemma}
 Let \(h:\R^n\rightarrow\R^m\).  If
  \[
     h(x+y)=h(x)h(y)
  \]
  then either \(h(x)=0\) or h(x)=e\textsuperscript{f(x)}, where \(f\) is an arbitrary
solution to Cauchy's equation \Eq{cauchyeqn}.
\label{lem:cauchyexp}
\end{lemma}

\begin{corollary}
Any solution to the functional equation of \Lem{cauchyexp} which is continuous and non-constant
is of the form
  \[
     h(x)=\exp(Cx),
  \]
  where \(C\) is a constant matrix and the \(\exp\) operator is element by element.
\end{corollary}

The following is just a restatement of Theorem 15.1 of \cite{aczel-dhombres89}, and
describes all solutions to the generalized Pexider equation
\Eq{generalized_pexider} over the general domain of Abelian groupoids.
\begin{theorem}
For any \(x,y\) in an Abelian groupoid, solutions to \Eq{generalized_pexider}
will satisfy one of:
\begin{enumerate}
\item If \(\varphi(x)=1\) for all \(x\), then \(\kappa(x)\) is an arbitrary
function; \(\psi(x)=\kappa(x)\); and \(k(x)=\kappa(x)+B\). Or;
\item if \(\varphi(x_0)\neq 1\) for some \(x_0\), then we have
\(C=\frac{\kappa(x_0)}{\varphi(x_0)-1}\); and
 \(\kappa(x)=C[\varphi(x)-1]\); and two sub-cases:
\begin{enumerate}
\item \(C=0\); \(\kappa(x)=0\); \(\varphi(x)\) arbitrary; \(k(x)=B\);
\(\psi(y)=B(1-\varphi(y))\); or
\item \(C\neq 0\); \(k(x)=C\varphi(x) + B\); \(\psi(x)=B(1-\varphi(x))\);
where \(\varphi(x)\) satisfies
\(\varphi(x+y)=\varphi(x)\varphi(y)\) (Cauchy's exponential
equation).
\end{enumerate}
\end{enumerate}
\label{thm:generalized_pexider}
\end{theorem}

If we restrict the domain under consideration to a real vector space,
then we can give explicit solutions to \Eq{generalized_pexider}, as follows:
\begin{corollary}
For any \(x,y\in\R^n\), solutions to \Eq{generalized_pexider} will satisfy one
of:
\begin{enumerate}
\item If \(\varphi(x)=1\) for all \(x\), then \(\kappa(x)=\psi(x)=Cx\) and \(k(x)=Cx+B\),
where \(B\in\R^m\). Or;
\item if \(\varphi(x_0)\neq 1\) for some \(x_0\), then we have
\(C=\frac{\kappa(x_0)}{\varphi(x_0)-1}\); and
 \(\kappa(x)=C[\varphi(x)-1]\); and two sub-cases:
\begin{enumerate}
\item \(C=0\); \(\kappa(x)=0\); \(\varphi(x)\) arbitrary; \(k(x)=B\);
\(\psi(y)=B(1-\varphi(y))\); or
\item \(C\neq 0\); \(k(x)=C\varphi(x) + B\); \(\psi(x)=B(1-\varphi(x))\);
\(\kappa(x)=Cx\); and one of:
\begin{enumerate}
\item \(\varphi(x)=0\);
\item \(\varphi(x)=\exp(Ax)\); or
\item \(\varphi(x)=\exp(f(x))\), \(f\) nowhere continuous.
\end{enumerate}
\end{enumerate}
\end{enumerate}
\label{cor:generalized_pexider}
\end{corollary}
\begin{proof}
 Just a specialization of \Thm{generalized_pexider} to the case in which domain
 is a real vector space, which then allows subsequent application of
\Lem{cauchyeqn} and  \Lem{cauchyexp}.
\end{proof}
\subsection{Proof of Theorem}
\label{sec:orgc71727d}
We're now in a position to prove our main result:
\begin{proof}
First, \Lem{xhomogeneous} establishes that \((\phi_j,a_j,b_j)\) in
\Eq{xlseparable} are all either logarithmic or positive homogeneous
of some degree \((-)\beta_j\).

In the logarithmic case the logarithm of demand for good \(j\) can be
 written \(\log f_j(\lambda p) = [-\log p_j + a_j(p)] + b_j(\lambda)\),
 which not only has expenditures \(\lambda\)-separable, but also
quantities \(\lambda\)-separable.  Then the main result of \textcite{ligon16:lambda_separable_quantities} applies, with \(\phi_j=\log\), yielding the result that
 \(\log(f_j(\lambda p))=\tilde\alpha_j-\beta_j\log (p_j\lambda)\). Let
 \(c_j=f_j(\lambda p)\) and solve for \(p_j\lambda\), obtaining
 \(p_j\lambda=\alpha_jc_j^{-1/\beta_j}\), where
 \(\alpha_j=e^{\tilde\alpha_j}\) must be positive.

In the homogeneous case, we have
 \[
    (p_jf_j(\lambda p))^{\beta_j} = a_j(p) + b_j(\lambda),
 \]

 or
  \[
    f_j(\lambda p)^{\beta_j} = p_j^{-\beta_j}a_j(p) + p_j^{-\beta_j}b_j(\lambda).
 \]
  \noindent This takes the form of the generalized Pexider equation
 \Eq{generalized_pexider}, with \(x=\log\lambda\) and \(y\) the vector
 \(\log p\), when the vector-valued function
 \(k(x+y)=[f_j(\exp(x+y))^{\beta_j}]\),
 \(h(y)=[a_j(\exp(y))e^{-\beta_jy_j}]\), \(g(x)=[b_j(\exp(x))]\), and
 \(\ell(y)=[e^{-\beta_jy_j}]\).  Now, we seek to apply
 \Cor{generalized_pexider}, which gives solutions to the system of
 functional equations \ref{eq:generalized_pexider}--\ref{eq:kappa2}.
 Part of this system is the function \(\varphi(y)\).  Using our
 knowledge that \(\ell_j(y)=e^{-\beta_jy_j}\) and \Eq{varphi}, it
 follows that in this equation the function
 \(\varphi(y)=\ell(y)=[e^{-\beta_jy_j}]\).  Now, consulting the
 different possible cases of \Cor{generalized_pexider} we see that
 with this solution of \(\varphi\) the only cases that can apply are the
 cases indicated by 2a and 2bii.  The former implies that \(f_j(\lambda
 p)^{\beta_j}\) is a constant, so that (to be consistent with the
 properties of Frischian demands) \(\beta_j=0\).  But then the function
 \(\phi_j\) isn't increasing, and the only solutions that are relevant
 to our problem are the solutions 2bii.  These imply that \(k(z)=A\exp(Cz) +
 B\), with \(A\), \(B\) and \(C\) constant matrices, and \(\varphi(y)=\exp(Cy)\).
 Thus \(C\) is a diagonal matrix, with diagonal elements \(-\beta_j\).  Then we have
\[
     k(z) = f_j(e^z)^{\beta_j}=\alpha_j e^{-\beta_jz} + \sigma_j,
\]
the last substituting \(A=\alpha_j\) and \(B=\sigma_j\) in this scalar equation.  With a change of variables \(z=\log p + \log\lambda\) we obtain \(f_j(\lambda
 p)^{\beta_j}=\alpha_j/(p_j\lambda)^{\beta_j} + \sigma_j\).  Noting that \(u_j(c)=p_j\lambda\) and solving for this gives us the solution for marginal utilities.
\end{proof}
\section{Details of Estimation}
\label{app:details_of_estimation}
Here we describe the steps involved in our estimation of the demands given by equation \Eq{estimating}.\footnote{Additional discussion of assumptions and methods is given in \textcite{ligon20estimating}, while code to compute estimates is provided by \textcite{ligon25:cfe_code}.\label{fn:code}}
  We assume that observed household characteristics \(z_{it}\) affect log expenditures linearly, and that all households within a given market-year in Uganda face the same relative prices.\footnote{We take a market to correspond to one of four regions.  There are eight years, resulting in a total of 32 different relative price vectors.  If \emph{levels} of prices differ within a region this will be reflected in marginal utilities of expenditure.}  The latter assumption allows us to estimate the influence of prices by using market-year dummy variables, while the former allows us to express the influence of household characteristics on demand for good \(j\) as \(\gamma_j' z_{it}\), where \(\gamma_j\) is a vector of parameters to be estimated, and where \(t\) is understood to index market-years.   Substitute \(w_{it}=-\log\lambda_{it}\), and let \(y^j_{it}=\log x^j_{it}\). Then \Eq{estimating} becomes
\begin{equation}
\label{eq:estimating_linear}
     y_{it}^j = a_{jt} + \gamma_j'z_{it} + \beta_jw_{it} + \epsilon_{it}^j.
\end{equation}
If we observed either \(\beta_j\) or \(w_{it}\) this would be an entirely standard linear regression problem.  We do not, making this what \textcite{hansen22:econometrics} calls a factor model with additional regressors.\footnote{Our estimation approach follows Hansen's procedure, though by assuming that \(w_{it}\) is orthogonal to \(z_{it}\) we avoid the need for iteration.}

We make four standard assumptions.  First, that \((z_{it},w_{it})\) are orthogonal to the error \(\epsilon_{it}^j\), the usual identifying assumption in a linear regression context.  Second, that \(w_{it}\) is orthogonal to \(z_{it}\).  While not necessary for identification \autocite[see][]{williams20}, this assumption simplifies  both the interpretation and estimation of \(w_{it}\).  Third, that errors \(\epsilon_{it}^j\) are homoskedastic across goods, though we allow clustering at the market-year level, so that \(\mbox{Var}(\epsilon_{it}^j)=\mbox{Var}(\epsilon_{it}^{k})\).  Both the second and third assumptions are testable, and can be relaxed. Finally, as with any factor analysis some normalization is required for identification; here it's convenient to take \(\mbox{Var}(w_{it})=1\).

Estimation is straight-forward, and proceeds in three steps:
\begin{enumerate}
\item Regress log expenditures on observed characteristics and time-market dummies
 \[
          y_{it}^j = a_{jt} + \gamma_j'z_{it} + e_{it}^j,
         \]
yielding estimates of the vector of the parameter vector \(\gamma\) and estimated residuals \(\hat{e}_{it}^j\).
\item The residuals \(\hat{e}_{it}^j\) provide an unbiased estimate of \(\beta_jw_{it} + \epsilon_{it}^j\).   Let \(\hat{e}_{it}\) denote the \(J\)-vector of residuals for household \(i\) at \(t\), and let \(\Sigma=\E e_{it}e_{it}'\) be its covariance matrix.  From \Eq{estimating_linear} and our assumptions above this implies
\[
                     \Sigma = \beta\beta' + \sigma I.
              \]
This is a classical factor analysis problem, and the (quasi-) maximum likelihood estimate of \(\beta\) is proportional to the first principal component of the sample covariance matrix \(\hat{\Sigma}\) \autocite{anderson-rubin56}.  Our normalization \(\mbox{Var}(w_{it})=1\) supplies the missing factor of proportionality.
\item Finally, we use a standard ``regression'' method \autocite{bartholomew-etal11} to estimate the \(w_{it}\).  We return to \Eq{estimating_linear}, but now treat the \(w_{it}\) as unknown parameters to be estimated in the regression
\[
                      y_{it}^j = a_{jt} + \gamma_j'z_{it} + \hat{\beta}_j w_{it} + \tilde{\epsilon}_{it}^j,
                 \]
where the difference from the original \Eq{estimating_linear} is that now we can include the estimated \(\hat{\beta}_j\) as (generated) regressors.
\end{enumerate}
\subsection{Observations}
\label{sec:orgef86eba}
\begin{itemize}
\item Of the three steps outlined above, two are least squares regressions, the properties of which will be familiar to economists.   The second step employs factor analysis---a widely used procedure in other fields but less common in economics.  A key consideration in factor analysis concerns the assumed structure of the disturbance covariance matrix across goods, say \(\Psi=\frac{1}{NT}\E\sum_{i,t}\epsilon_{it}\epsilon_{it}'\).  Here we assume a form of homoskedasticity, with  \(\Psi=\sigma I\), in which case the first principal component of \(\hat\Sigma\) provides an unbiased and efficient estimate of  \(\beta\).   Some alternative assumptions:
\begin{itemize}
\item If \(\Psi\) is diagonal, the classical approach \autocite{anderson-rubin56} uses maximum likelihood estimation assuming normally distributed disturbances. Even when this distributional assumption isn't satisfied, the (quasi-) maximum likelihood estimator retains good properties and can be implemented through a simple iterative procedure.
\item If \(\Psi\) is unrestricted, \textcite{bai03} provides conditions under which our principal components estimator still possesses desirable properties. The caveat is that the argument is asymptotic, requiring both \(J\) and \(TN\) to approach infinity. In our application, while \(TN\) is reasonably large, \(J\) is measured only in dozens.
\item \textcite{connor-korajczyk86} establish consistency of our principal components estimator under a weaker sufficient and \autocite{bai03} necessary condition than our assumption that \(\Psi=\sigma I\).  Instead, they require only an asymptotic version of this, requiring that errors across goods be uncorrelated and homoskedastic as \(NT\) grows large, but holding the number of goods \(J\) fixed.
\item \textcite{williams20} establishes identification results for linear factor models for fixed \(J\) and very modest restrictions on \(\Psi\). However, to our knowledge, no practical estimators have yet been developed for this case.
\end{itemize}
\item We interpret \(\epsilon_{it}^j\) as possibly being \emph{unobserved} household characteristics that affect demand, such as an idiosyncratic desire for groundnuts.  But if ostensibly \emph{observed} characteristics such as household size are measured with error, then that measurement error will also show up in the residual, with familiar effects on estimates of  the \(\gamma_j\) (e.g., attenuation bias).   This will change the covariance matrix \(\Psi\), but as the references above suggest, estimates of \((\beta,w)\) will still have desirable properties under fairly general conditions.\label{ref:z_error}
\item With a consistent estimator of \(\Sigma\), we can estimate \(\beta\). Notably, it is possible to estimate \(\Sigma\) consistently even when some food expenditure data is missing, as is the case in our application \autocite{little21}.
\end{itemize}
\section{For Online Publication: Direct Evidence on the Effects of Covariate Shocks\hfill{}\textsc{online}}
\label{app:effect_of_shocks}
\subsection{Introduction}
\label{sec:org30ba4fd}
In this appendix we draw on sources of data not needed in the main analysis to try to directly explore the effects of covariate shocks on a variety of outcomes, with the aim of tracing whether and how such shocks affect production, income, and consumption; what mechanisms households use to cope with these shocks; the effects of covariate shocks on local prices (both levels and relative); and finally on food consumption for the household. We provide direct evidence that: (i) Preferences are nonhomothetic (expenditure shares vary with income and prices); (ii) Covariate shocks are indeed spatially correlated, in the sense that variation between clusters is greater than would be expected by random chance; (iii) People who reported having experienced a shock also tended to report declines in agricultural production and income; (iv) People reported coping with covariate shocks by reducing their consumption, using savings, or increasing labor, while they reported coping with idiosyncratic shocks by relying on help from relatives and friends;  (v) Farmgate prices of locally produced goods increase significantly when a covariate shock is realized; (vi) Covariate shocks are associated with differences in the relative market prices consumers face; and (v) Covariate shocks of drought, floods, or pests are all associated with statistically significant changes in diet, and in particular a decreased demand for food obtained from animal sources such as dairy products and different kinds of meat.
\subsection{Preferences are nonhomothetic}
\label{sec:org8b65133}

\label{r2:nonhomothetic}
With homothetic preferences the composition of the consumption portfolio is irrelevant, but for our approach the composition is central.  Can we see direct evidence in the Ugandan expenditure data for or against homotheticity?
\Fig{agg_shares_and_mean_shares} uses the fact that homothetic preference structures imply that \emph{aggregate} expenditure shares (of the sort that appear in national income and product accounts) will be the same as \emph{average} expenditure shares.  The figure takes logs of each of these, and then differences.  It is apparent that these shares are not the same at all, with goods to the right (e.g., beer) playing a larger role in the consumption portfolios of households with larger budgets, and goods to the left (e.g., sorghum) playing a more important role for poorer households.

\begin{figure}[htbp]
\centering
\includegraphics[height=.9\textheight]{./Figures/agg_shares_and_mean_shares.png}
\caption{\label{fig:agg_shares_and_mean_shares}Log of aggregate shares minus log of mean shares for different years (ordered by aggregate share averaged across years), with 95\% confidence intervals.}
\end{figure}
\subsection{Incidence and covariance of shocks}
\label{sec:org16f84bd}
\subsubsection*{Incidence of Shocks}
\label{sec:orgecfda64}
The data we have from the LSMS surveys includes self-reported data on a variety of ``shocks'' the household may have experienced \autocite[discuss the collection of this sort of data in a variety of different household surveys]{heltberg-etal15}, generally elicited using the prompt ``Did you experience [SHOCK] in the last twelve months?''\footnote{There is some variation in the elicitation in different rounds, and the first 2005-06 round in particular uses a longer reporting period.}  Where the answer is ``Yes'', the respondent is asked about the timing of the shock (in what month did the shock occur, and for how long did it last).   There is some modest variation across rounds in the language used to describe different sorts of shocks (see Appendix \ref{app:shocks}), but one can distinguish two classes.  The  first is idiosyncratic, shocks which directly involve a particular household.  Frequently reported idiosyncratic shocks include health issues (serious illness or accident), thefts of property, and death (death of ``income earners'' is reported separately from the death of other household members).   The second is covariate, shocks which seem likely to affect many households within a local area, though not necessarily equally.  Frequently reported covariate shocks are drought (or ``irregular rains''), agricultural pests, floods, and adverse agricultural prices (unusually expensive inputs or unusually low prices for output).  Table \ref{tab:shocks_by_year} reports the incidence of these shocks across different rounds.

\begin{table}[htbp]
\caption{\label{tab:shocks_by_year}Reported incidence of different kinds of shocks by year.}
\centering
\begin{tabular}{llllllllll}
\toprule
Shock & 2005 & 2009 & 2010 & 2011 & 2013 & 2015 & 2018 & 2019 & Pooled\\
\midrule
Health & 2\% & 12\% & 11\% & 5\% & 4\% & 2\% & 5\% & 6\% & 3\%\\
Theft & 11\% & 7\% & 3\% & 1\% & 2\% & 1\% & 2\% & 2\% & 2\%\\
Death (non-earner) & 13\% & 2\% & 2\% & 1\% & 2\% & 1\% & 1\% & 1\% & 1\%\\
Death of earner & 3\% & 0\% & 0\% & 0\% & 0\% & 0\% & 0\% & 0\% & 0\%\\
Drought & 39\% & 45\% & 26\% & 19\% & 29\% & 18\% & 22\% & 17\% & 13\%\\
Floods & 13\% & 2\% & 3\% & 5\% & 3\% & 1\% & 2\% & 3\% & 2\%\\
Pests & 15\% & 7\% & 2\% & 3\% & 2\% & 1\% & 4\% & 2\% & 2\%\\
Prices & 2\% & 3\% & 2\% & 2\% & 2\% & 0\% & 2\% & 0\% & 1\%\\
\midrule
\(N\) & 3122 & 2974 & 2685 & 2843 & 3117 & 3305 & 3241 & 3076 & 24363\\
\bottomrule
\end{tabular}
\end{table}

To get a notion of frequency from \Tab{shocks_by_year}, there are about 2800 households observed per year.  Drought is by far the most frequently reported shock.   Meteorologists regard  2005--08 as a period of major drought for Uganda, with 2010--11 and 2014--15 periods of minor drought \autocite{byakatonda-etal21}, consistent with the household reports in \Tab{shocks_by_year}.  Health shocks (both illness and accidents) are the second most frequently reported shocks, followed by pests (which includes both crop pests and livestock disease), floods, death of non-earner, theft, and adverse changes in agricultural prices (i.e., increases in the costs of inputs; decreases in the value of outputs).
\subsubsection*{Covariate shocks are spatially correlated}
\label{sec:covariate_shocks}
We use several distinct sorts of evidence to establish that the ``covariate'' shocks we have data on do indeed covary.     First, we test for geographical ``clustering'' of reported shocks within the Ugandan LSMS data.  Second, we consider similar data from other LSMS surveys in Africa (eight countries in all), and construct maps of the reported incidence of different shocks, which we think provides informal but compelling visual evidence of spatial covariation.  Third, we use these same data to estimate spatial autocorrelation functions, and establish that below a certain distance these estimated autocorrelations functions would be very unlikely to be observed under the null of no spatial correlation.

The sampling scheme for the LSMS surveys involve randomly selecting geographical clusters (e.g., villages or parishes), and then randomly selecting households from within those clusters.   One would then suppose that the realizations of ``covariate'' shocks ought to involve significant variation ``between'' clusters.  Here we test that supposition by decomposing the variance of reports of different shocks in each year into variation attributable to variation in reporting ``within'' clusters and ``between'' clusters.

\begin{table}
\begin{threeparttable}
\caption{Proportion of total variance in shock reports attributable to variation between clusters, by year.  Figures in parentheses are \(p\)-values\tnote{$\dagger$} associated with the null hypothesis that the ``between'' variancc is no greater than the pooled variance.}
\label{tab:between_variance}

{\small
\begin{tabular}{r|D{.}{.}{-1}@{}D{.}{.}{-1}@{}D{.}{.}{-1}@{}D{.}{.}{-1}@{}|D{.}{.}{-1}@{}D{.}{.}{-1}@{}D{.}{.}{-1}@{}D{.}{.}{-1}}
\toprule
 &  &  &  &  &  & \hd{Death} & \hd{Death} & \\
Round & \hd{Drought} & \hd{Floods} & \hd{Pests} & \hd{Prices} & \hd{Health} & \hd{(non-earner)} & \hd{(earner)} & \hd{Theft}\\
\midrule
2005-06 & 0.11\ts{***} & 0.09\ts{***} & 0.08\ts{**} & 0.06 & 0.06 & 0.08\ts{**} & 0.07 & 0.07\\
 & (0.00) & (0.00) & (0.01) & (0.84) & (0.70) & (0.01) & (0.27) & (0.15)\\
2009-10 & 0.18\ts{***} & 0.12\ts{***} & 0.11\ts{***} & 0.11\ts{***} & 0.08 & 0.09\ts{*} & 0.07 & 0.09\\
 & (0.00) & (0.00) & (0.00) & (0.00) & (0.45) & (0.09) & (0.75) & (0.12)\\
2010-11 & 0.21\ts{***} & 0.17\ts{***} & 0.13\ts{***} & 0.09 & 0.11\ts{**} & 0.09 & 0.10 & 0.10\\
 & (0.00) & (0.00) & (0.00) & (0.67) & (0.02) & (0.75) & (0.28) & (0.27)\\
2011-12 & 0.19\ts{***} & 0.15\ts{***} & 0.12\ts{***} & 0.12\ts{***} & 0.09 & 0.10 & 0.12\ts{**} & 0.11\ts{**}\\
 & (0.00) & (0.00) & (0.01) & (0.01) & (0.32) & (0.14) & (0.02) & (0.05)\\
2013-14 & 0.34\ts{***} & 0.27\ts{**} & 0.19 & 0.17 & 0.23 & 0.22 & 0.16 & 0.19\\
 & (0.00) & (0.02) & (0.58) & (0.81) & (0.13) & (0.34) & (0.77) & (0.62)\\
2015-16 & 0.46\ts{***} & 0.29 & 0.19 & 0.25 & 0.23 & 0.23 & 0.17 & 0.29\\
 & (0.00) & (0.16) & (0.90) & (0.48) & (0.67) & (0.60) & (0.81) & (0.22)\\
2018-19 & 0.32\ts{***} & 0.27\ts{*} & 0.20 & 0.20 & 0.24 & 0.23 & 0.10 & 0.23\\
 & (0.00) & (0.07) & (0.85) & (0.72) & (0.18) & (0.42) & (0.95) & (0.42)\\
2019-20 & 0.29\ts{***} & 0.28 & 0.19 & 0.21 & 0.26 & 0.30 & 0.19 & 0.29\\
 & (0.00) & (0.10) & (0.93) & (0.73) & (0.23) & (0.13) & (0.66) & (0.13)\\
\bottomrule
\end{tabular}}

\begin{tablenotes}
\item[$\dagger$]{All \(p\)-values computed using randomized permutation tests, with 1000 draws.}
 \item[*]{\(p<0.10\).}
 \item[**]{\(p<0.05\).}
 \item[***]{\(p<0.01\).}
\item[$N$]{Total observations are 28,588.}
\item[$T$]{All regressions involve 8 rounds, 4 markets, and an average of 550 clusters per round.}
\item[$H$]{A total of 5771 distinct households (an average of 3045 households per round).}
\end{tablenotes}
\end{threeparttable}
\end{table}

\Tab{between_variance} provides estimates of the proportion of variance which is ``between'', along with the \(p\)-values associated with the one-sided null hypothesis that this variance is no greater than would be expected if there was no clustering.  Several features of this table are worth mentioning.  First, there is statistically significant clustering (between variation) for drought in every year, for floods in most years, for pests in four of eight years, and for prices in two of eight years.  Some of these failures to reject the null hypothesis may simply be due to the shock being reported infrequently; for example, there were only 12 reports of adverse agricultural prices in total in 2015-16, so the power of the test in that year is extremely low.   Second, while for covariate shocks there's most often unanimity within a cluster (either no one says  there's been a covariate shock or everyone says there has been), so that within-variance is zero for these clusters, there are also many clusters in which some but not all households report experiencing the shock.  Third, statistically significant clustering is observed less frequently for the shocks we've categorized as ``idiosyncratic,''  however, we still reject no clustering somewhat more frequently than we would expect under the null.  Three of the four reported ``idiosyncratic'' shocks involve sickness or death, and it's reasonable to suppose that a spatially correlated spread of disease may sometimes be implicated in these reports.

While there's important variation across years in the proportion of households that report drought, we would like to know that the data are consistent with those shocks being covariate, in the sense that if a person in one cluster reports a particular sort of shock, then another nearby person is likely to report the same.  \Fig{drought_reports} documents all the drought reports across the eight different LSMS-ISA countries over all the years (2005--20) for which we have data, while \Fig{flood_reports} and \Fig{pests_reports} do the same for floods and pests, respectively.\footnote{For our examination of data across African countries we rely on the dataset described by \textcite{bentze-wollburg24}.  Note that this dataset categorizes shocks somewhat differently than we do in our analysis of data from Uganda, and evidently uses a narrower definition of ``floods'' than we do.}  Dots correspond to sample clusters, and the ``redder'' the dot, the higher the proportion of households which reported drought.
\begin{figure}[htbp]
\centering
\includegraphics[width=.9\linewidth]{../Figures/LSMS-ISA/drought_incidence.png}
\caption{\label{fig:drought_reports}Drought reports in the LSMS-ISA data across Africa.  Each dot corresponds to a sample cluster; the ``bluer'' the dot the higher the proportion of households in the cluster which reported drought.  Note that observations from later sample years often partly obscure data from earlier years.}
\end{figure}

Casual observation of  \Fig{drought_reports} suggests not only that there are many dark dots (high proportions of households within a cluster agreeing that there's drought), but also that darker dots tend to be close to each other, consistent with the notion that these reports really are reflecting covariate shocks.

\begin{figure}[htbp]
\centering
\includegraphics[width=.9\linewidth]{../Figures/LSMS-ISA/flood_incidence.png}
\caption{\label{fig:flood_reports}Flood reports in the LSMS-ISA data across Africa.  Each dot corresponds to a sample cluster; the ``bluer'' the dot the higher the proportion of households in the cluster which reported floods.  Note that observations from later sample years often partly obscure data from earlier years.}
\end{figure}

\begin{figure}[htbp]
\centering
\includegraphics[width=.9\linewidth]{../Figures/LSMS-ISA/pests_incidence.png}
\caption{\label{fig:pests_reports}Pest reports in the LSMS-ISA data across Africa.  Each dot corresponds to a sample cluster; the ``bluer'' the dot the higher the proportion of households in the cluster which reported pests.  Note that observations from later sample years often partly obscure data from earlier years.}
\end{figure}


We've seen evidence of correlation in shock reports within clusters (villages or parishes).  Is there also correlation across nearby clusters?  We have data on cluster location, which we use to we take a more formal approach to testing the hypothesis of no spatial dependence across clusters in shock reports.  We use the data from shock reports across Africa described above, and then follow \textcite{conley-ligon02} in computing non-parametric spatial autocorrelation functions for reports of drought, floods, or pests (adverse price shocks are not reported in this cross-country dataset).  This yields pointwise estimates of the correlation of shock reports at different distances; we're interested in whether or not the observed spatial autocorrelations could have been generated by chance under the null hypothesis of no autocorrelation.

There are two important sources of error one should be aware of.  First, the public LSMS data intentionally obscures location in order to protect respondent identity by mapping the location of each household to a single cluster ``point'', and then randomly shifting the location across space.   However, the rules for this random displacement are clear and well-specified \autocite{burgert-etal13}, so we are able to take the effects of this displacement on estimation into account in constructing our tests.  The second source of error is more profound---the underlying statistical model assumes that while \emph{distance} matters for correlation, \emph{direction} does not (i.e., an assumption of isotropy).   At very large scales this is obviously wrong in our data---going in the direction of the Sahara matters for drought.  However, the distances we consider are on the order of hundreds of kilometers, so very large-scale features such as the Sahara are in some sense far away from most of the clusters in our data.   But isotropy may be a bad assumption even at smaller scales for some shocks such as floods, where the probability of nearby flooding is much greater if one follows the direction of a watercourse than otherwise.  Fortunately, \textcite{conley-molinari07} provide evidence that hypothesis tests using our non-parametric estimator are quite robust to these sorts measurement error in location.

\begin{figure}[htbp]
\centering
\includegraphics[width=.9\linewidth]{../Figures/LSMS-ISA/shock_spacfs.png}
\caption{\label{fig:shock_spacfs}Estimated spatial autocorrelation functions for shock reports.  Dotted area indicates a point-wise 95\% ``acceptance'' region of the null hypothesis of no correlation at a given distance.}
\end{figure}  

We proceed by using provided data on cluster locations, then computing the great-circle distance between each pair of clusters in each year.  We then use the non-parametric spatial autocovariance estimator of \textcite{conley99}, using a rectangular kernel with a bandwidth distance of roughly 30 kilometers.   For each year we obtain an autocorrelation function by dividing the autocovariance function at each distance \(d\) by the value of the function at a distance of zero (i.e., by the variance of cluster incidence per year).  We use a sort of augmented randomized permutation inference (with 1000 draws) to construct 95\% acceptance regions for the estimated functions, and construct an average across years, weighted by the number of observations per year.

The augmentation of our randomized permutation deserves some remark---this is intended to take into account the intentional random displacement of geographical coordinates imposed on the public data \autocite{burgert-etal13}.  There are two key features of this displacement for us: first, one percent of all locations are moved some distance in a random direction, where that distance is drawn from a uniform distribution on \([0,100]\) kilometers.  Then \emph{all} locations are similarly displaced, but only up to five kilometers.  Under the null of no autocorrelation location doesn't matter, so we can treat the observed locations as though they were measured without error, and apply the displacement algorithm to each permutation of locations to discover the effect on inference.

Results are shown in Figure \ref{fig:shock_spacfs}.  In our randomized permutation setting distances at which the estimated spatial autocorrelation function lies above the acceptance region are distances at which the \(p\)-value associated with the null hypothesis is less 2.5\% (for a one-tailed test).  Here we see evidence of positive spatial autocorrelation for all three shocks.    This evidence is least pronounced for floods, where significant autocorrelation is observed at distances of about 30--60 km.  Here the assumption of isotropy may be playing an important role, as two nearby clusters may well be at different elevations.  There is evidence of positive autocorrelation at a much larger range of distances for both drought (0--400 km) and pests (0--470 km).
\subsection{Covariate shocks \& reported declines in production, income, consumption, and assets}
\label{sec:orgbfce78a}
The LSMS survey module on shocks elicits subjective reports on whether the shock resulted in a decline in the outcomes production, income, assets, or consumption.  Table \ref{tab:shock_affected} reports the proportion of people who, having reported that they'd experienced a particular sort of shock (such as drought) in the last twelve months, \emph{also} responded affirmatively to the question: ``As a result of the [SHOCK] was there a decline in your household's [OUTCOME]?''

\begin{table}[htbp]
\caption{\label{tab:shock_affected}Subjective reports as to whether shock resulted in a decline in particular outcomes.  These reports are conditional on the respondent reporting that they'd experienced the shock in the last 12 months.}
\centering
\begin{tabular}{lllll}
\toprule
Shock Affected & Drought & Floods & Pests & Prices\\
\midrule
Production & \(0.92\) & \(0.89\) & \(0.73\) & \(0.47\)\\
Income & \(0.80\) & \(0.67\) & \(0.77\) & \(0.84\)\\
Consumption & \(0.50\) & \(0.41\) & \(0.37\) & \(0.44\)\\
Assets & \(0.25\) & \(0.25\) & \(0.35\) & \(0.27\)\\
\bottomrule
\end{tabular}
\end{table}

Thus, of households who reported drought in the last twelve months, 92\% asserted that the drought resulted in a decline in production, and 80\% reported a decline in income.    The logic of these responses seems fairly clear: a drought will directly affect agricultural production, and so \emph{ceteris paribus} a farmer will have less income from marketing his smaller output.

The remaining two outcomes seem less direct.   For a shock seems likely to affect consumption only indirectly, by affecting available resources (such as income or assets) which might be used to finance consumption. And assets could be affected in two different ways: via reduced resources (leading to lower rates of investment or sale of assets), or by directly damaging physical assets.
\subsection{Covariate shocks increase local farmgate prices}
\label{app:shocks_and_farmgate_prices}
Respondents to the LSMS surveys in Uganda report farmgate prices for goods they produce, and so we can see how these prices respond to covariate shocks.    One concern one might have is that respondents who report shocks might also be inclined to report higher prices.  We address this by calculating mean prices for every year-market.   For each good we then regress changes (over time) in log prices on the proportion of sample households in every market-year that report different aggregate shocks and on a complete set of year effects.  The resulting coefficients can be interpreted as the elasticities (of prices to proportions of reporting households).

\begin{figure}[htbp]
\centering
\includegraphics[width=.9\linewidth]{./Figures/shocks_and_farmgate_prices.png}
\caption{\label{fig:shocks_and_farmgate_prices}Shocks and Farmgate Prices. Point estimates from a regression of changes in mean year-market (farmgate) log prices on the proportion of households reporting particular covariate shocks in a given year-market; includes year effects.  Bars indicate 95\% confidence intervals.  Ordered by average magnitude of coefficient across shocks.}
\end{figure}

We observe changes in log mean prices\footnote{Using log median prices delivers broadly similar results.} for 20 goods over four markets and over seven periods, so the sample is small (just 28 observations of price changes for each  good).  Nevertheless, results from this exercise are shown in \Fig{shocks_and_farmgate_prices} for the four covariate shocks ``drought'', ``floods'', ``pests'', and adverse ``prices''.  We have farmgate prices in every market-year for 20 different goods.   Point estimates on ``drought'', ``pests'', and ``prices'' are positive for all goods; for ``floods'' they are positive in 9 of 20 cases.   Though a \(t\)-test would reject the null of zero effect in only a handful of cases (12,0,7, and 3, respectively), this is unsurprising given our small sample.  A more appropriate test of whether shocks positively  affect farmgate prices in general is the non-parametric sign test.  Under the null of no effect each point estimate is equally likely  to be positive or negative; the \(p\) values associated with this null hypothesis are 0.41 in the case of floods, and less than 0.000001 for drought, pests, or prices.
\subsection{Mechanisms used to cope with shocks}
\label{sec:org5a8a1d1}
\begin{table}[htbp]
\caption{\label{tab:how_coped_covariate}How did households cope with different covariate shocks?  These reports are conditional on the respondent reporting that they'd experienced the shock in the last 12 months.  Respondents could report up to three different coping mechanisms, so columns do not sum to one.}
\centering
\begin{tabular}{llllll}
\toprule
 & Drought & Floods & Pests & Prices & Any\\
How Coped? &  &  &  &  & \\
\midrule
Reduced consumption & \(0.51\) & \(0.35\) & \(0.24\) & \(0.21\) & \(0.44\)\\
Used savings & \(0.45\) & \(0.47\) & \(0.38\) & \(0.37\) & \(0.44\)\\
Reduced leisure & \(0.31\) & \(0.26\) & \(0.14\) & \(0.18\) & \(0.28\)\\
Help from relatives/friends & \(0.21\) & \(0.20\) & \(0.14\) & \(0.10\) & \(0.20\)\\
Nothing & \(0.16\) & \(0.15\) & \(0.10\) & \(0.15\) & \(0.15\)\\
Other & \(0.10\) & \(0.12\) & \(0.24\) & \(0.35\) & \(0.13\)\\
Changed cropping practices & \(0.10\) & \(0.08\) & \(0.21\) & \(0.16\) & \(0.11\)\\
Sold assets & \(0.08\) & \(0.07\) & \(0.10\) & \(0.06\) & \(0.08\)\\
Used credit & \(0.08\) & \(0.07\) & \(0.05\) & \(0.07\) & \(0.07\)\\
Help from local government & \(0.03\) & \(0.03\) & \(0.06\) & \(0.01\) & \(0.03\)\\
Reduced investments in human capital & \(0.03\) & \(0.01\) & \(0.02\) & \(0.02\) & \(0.03\)\\
Sent children elsewhere & \(0.01\) & \(0.02\) & \(0.01\) & \(0.00\) & \(0.01\)\\
Migrated & \(0.01\) & \(0.01\) & \(0.01\) & \(0.00\) & \(0.01\)\\
Rented out assets & \(0.00\) & \(0.00\) & \(0.01\) & \(0.00\) & \(0.00\)\\
\midrule
Average \# Coping Methods & \(2.09\) & \(1.84\) & \(1.69\) & \(1.70\) & \(1.99\)\\
\bottomrule
\end{tabular}
\end{table}

\begin{table}[htbp]
\caption{\label{tab:how_coped_idiosyncratic}How did households cope with different idiosyncratic shocks?  These reports are conditional on the respondent reporting that they'd experienced the shock in the last 12 months.  Respondents could report up to three different coping mechanisms, so columns do not sum to one.}
\centering
\begin{tabular}{llllll}
\toprule
 &  &  & Death & Death & \\
How Coped? & Health & Theft & (non-earner) & (earner) & Any\\
\midrule
Help from relatives/friends & \(0.58\) & \(0.20\) & \(0.74\) & \(0.77\) & \(0.52\)\\
Used savings & \(0.47\) & \(0.34\) & \(0.38\) & \(0.29\) & \(0.40\)\\
Other & \(0.15\) & \(0.28\) & \(0.10\) & \(0.06\) & \(0.17\)\\
Reduced consumption & \(0.17\) & \(0.13\) & \(0.06\) & \(0.20\) & \(0.14\)\\
Nothing & \(0.12\) & \(0.17\) & \(0.12\) & \(0.12\) & \(0.13\)\\
Sold assets & \(0.17\) & \(0.04\) & \(0.08\) & \(0.14\) & \(0.11\)\\
Used credit & \(0.13\) & \(0.06\) & \(0.08\) & \(0.07\) & \(0.09\)\\
Reduced leisure & \(0.10\) & \(0.07\) & \(0.03\) & \(0.17\) & \(0.08\)\\
Help from local government & \(0.03\) & \(0.06\) & \(0.02\) & \(0.03\) & \(0.04\)\\
Reduced investments in human capital & \(0.03\) & \(0.02\) & \(0.02\) & \(0.04\) & \(0.03\)\\
Sent children elsewhere & \(0.02\) & \(0.01\) & \(0.02\) & \(0.04\) & \(0.02\)\\
Migrated & \(0.01\) & \(0.01\) & \(0.01\) & \(0.02\) & \(0.01\)\\
Rented out assets & \(0.01\) & \(0.00\) & \(0.01\) & \(0.01\) & \(0.01\)\\
Changed cropping practices & \(0.01\) & \(0.01\) & \(0.00\) & \(0.01\) & \(0.01\)\\
\midrule
Average \# Coping Methods & \(2.00\) & \(1.41\) & \(1.66\) & \(1.98\) & \(1.76\)\\
\bottomrule
\end{tabular}
\end{table}


For each shock they reported, respondents were asked how they coped with the shock, and offered up to three responses.  Table \ref{tab:how_coped_covariate}\footnote{See Table \ref{howcoped_labels} for details of how responses are coded.} indicates the frequency with which different coping strategies were reported for covariate shocks.  For example, 51\% of respondents who reported experiencing drought said that they coped by reducing consumption, 45\% said that they used savings, and 31\% said they reduced leisure.  All of these mechanisms can be thought of as forms of self-insurance.  ``Help from relatives/friends'' is also commonly invoked (21\% of households who experienced one of these covariate shocks reported such help).  It should not be surprising that this kind of aid from friends and family is the dominant mechanism for idiosyncratic shocks, reported by 52\% (Table \ref{tab:how_coped_idiosyncratic}) of those who reported experiencing any of the idiosyncratic shocks found in Table \ref{tab:shocks_by_year}---a feature of covariate shocks is that others are likely to also experience them at the same time.  And this risk-sharing seems to be fairly effective in addressing idiosyncratic shocks, as fewer than 14\% of households who report idiosyncratic shocks say they reduced consumption in response.
\subsection{Covariate shocks and relative prices}
\label{app:shocks_and_relative_prices}
If a covariate shock has a proportionate effect on all prices, then this would be equivalent to a reduction in income, with no change in relative prices.   With nonhomothetic preferences this would still result in changes in the composition of expenditures, but these changes would be entirely due to changes in consumers' MUEs.

However, the kinds of shocks we consider seem likely to change \emph{relative} prices.  The exact nature of these changes will be complicated, depending not only on the effects of the shock on the supply of locally produced goods, but on income and price elasticities for all goods.   For example, a drought in an region where cultivation of maize and beans is important might increase bean prices more than maize, as maize tends to be traded over longer distances.  But while a drought might be expected to increase local bean prices (via a shift in local supply), it might also reduce local demand for other goods via an income effect.  In fact, if local production is dominated by the production of only a few goods, then we would expect a shock which affected the local supply of those goods to be associated with higher prices for those goods, while for other goods we would tend to expect income effects to reduce demand, causing prices to be lower than otherwise.\footnote{However, this simple story is complicated by the possibilities of substitution and inferior goods---for example, an increase in local bean prices might lead to an increase in demand for a substitute such as matoke, or the income effect might result in an increase in demand in an apparently inferior good such as millet.}

\begin{figure}[htb]
\centering
\includegraphics[width=.9\linewidth]{../Figures/Uganda/shocks_and_relative_prices.png}
\caption{\label{fig:shocks_and_relative_prices}Histograms of \(t\)-statistics testing hypothesis that covariate shocks have no effect on prices, by good (across market-years).  Shaded areas are 95\% acceptance regions.  Goods are those for which we estimate demands, listed in \Fig{beta_estimates}.}
\end{figure}

Here we wish to test whether the realization of a market level (regional) covariate shock leads to changes in relative prices.   We earlier looked at the effect of covariate shocks on changes in farmgate prices of goods which are locally produced.  Because our focus is in part on income effects here we look instead at reported market prices, which encompasses the larger set of goods we use for estimating MUEs, and which are the relevant prices for consumers.

Our approach here is to compute the deviation of mean log price of each good in each market-year from the average country-wide log price.   We then divide the sample of prices according to whether households report a particular shock versus households that don't report the shock.   We then simply construct a \(t\)-test of the null hypothesis that average log prices are the same across these two samples.

\Fig{shocks_and_relative_prices} shows the result of these tests for the four covariate shocks we've considered to this point.  Under the null hypothesis of the shock not affecting relative prices we'd expect five percent of computed \(t\) statistics to fall outside the acceptance region \((-1.96,1.96)\).  In fact, 82\%, 74\%, 44\%, and 53\% of the tests (respectively) reject equality.  Further, as suggested above, a negative shock tends to lead to \emph{lower} prices for most goods.
\subsection{Quantities of food demanded}
\label{sec:food_quantities}
How does diet respond to shocks?  We consider the effects of different covariate shocks on the consumption of different food items by regressing different outcomes on the proportion of households in each market-year who reported a particular shock, along with a complete set of time effects, so that the identifying variation is across market regions within a year.

We consider three related outcomes, all related to what we'll call ``food acquired'': the sum of all food purchased, consumed out of own production, received as a gift, or consumed outside the household by any household member during the last week.  The first outcome is simply quantities of food acquired, including ``zeros'' for food items that the household didn't report at all.  To handle differences in units and scale quantities are standardized.  Second, we consider changes in diet on the `extensive' margin, using an indicator for whether the household reported acquiring any positive amount of food on reports of shocks; one can thus interpret estimated shock coefficients as the probability of consuming a given good conditional on the shock.  This set of regressions is of some particular interest because it can be regarded as a  measure of how dietary diversity responds to covariate shocks.  Third, where amounts are positive we use the logarithm of quantities; the resulting coefficients are interpretable as elasticities related to the `intensive' margin.

Results are presented in Figures \ref{fig:shocks_and_level_food_quantities}, \ref{fig:shocks_and_positive_food_quantities}, and \ref{fig:shocks_and_log_food_quantities}.  These three figures report estimated coefficients associated with different shocks, along with horizontal bars that indicate 95\% confidence intervals. Goods are ordered according to magnitude of coefficients averaged across different shock regressions.

\begin{figure}[htbp]
\centering
\includegraphics[width=.9\linewidth]{../Figures/Uganda/shocks_and_level_food_quantities.png}
\caption{\label{fig:shocks_and_level_food_quantities}Food Quantities and Shocks.  Estimated coefficients from regression of standardized quantities of different food items on the share of households in a given market-year reporting given covariate shock (year effects included).  Bars indicate 95\% confidence intervals.  Goods are ordered according to magnitude of coefficients averaged across different shock regressions.}
\end{figure}

We first consider the quantity regression; coefficients are reported in Figure \ref{fig:shocks_and_level_food_quantities}.   Several points seem worth mentioning.  First, one might entertain the hypothesis that a covariate shock might reduce consumption of all kinds of food.  This idea is consistent with the logic of homothetic preferences such as the CRRA, where consumption scales linearly with  budget and all income elasticities are unity.  But we've already seen evidence that there are large differences in the relative income elasticities of different kinds of food.  Figure \ref{fig:shocks_and_level_food_quantities} suggests that the principal effect of a covariate shock is to change the composition of the food bundle, rather than reducing quantities of all foods--just looking at food items, average consumption of roughly half increases in response to shock.  And as with prices, patterns of substitution seem to be very important.  For example, average sim sim (sesame) consumption increases dramatically for all three covariate shocks reported in the figure.   From our estimates of the demand system (see Figure \ref{fig:beta_estimates}) sim sim is among the less income elastic goods, so we might expect its role in the diet to expand when income is low.  But other goods are even less income elastic, such as cassava. It appears that sim sim provides a protein-rich complement to cassava and sorghum in some staple dishes, substituting for meat \autocite{wanjala-etal16}, or being processed to substitute for other cooking oils (one of the more income elastic goods); collectively these patterns of complementarity and substitution may account for the large increase in sim sim consumption observed when there's a covariate shock.
\begin{figure}[htbp]
\centering
\includegraphics[width=.9\linewidth]{../Figures/Uganda/shocks_and_positive_food_quantities.png}
\caption{\label{fig:shocks_and_positive_food_quantities}Positive Quantities and Shocks.  Estimated coefficients from regression of an indicator of positive quantities of different food items on the share of households in a given market-year reporting given covariate shock (year effects included).  Bars indicate 95\% confidence intervals. Goods are ordered according to magnitude of coefficients averaged across different shock regressions.}
\end{figure}

We next try to decompose the average dietary changes observed in Figure \ref{fig:shocks_and_level_food_quantities} into changes on the extensive (Figure \ref{fig:shocks_and_positive_food_quantities}) and intensive (Figure \ref{fig:shocks_and_log_food_quantities}) margins.   Figure \ref{fig:shocks_and_positive_food_quantities} can be thought of providing estimates of the change in the share of households acquiring a particular food conditional on the proportion of households within its market region reporting a given shock.  So, for example, shocks are associated with a larger share of households consuming ``other vegetables'', sorghum, or sim sim, and with a dramatically smaller share consuming matoke (cooking bananas, a favored staple), fresh milk, or ``other fruit.''

We saw in Figure \ref{fig:shocks_and_level_food_quantities} that the number of foods that increased in quantity in response to a covariate shock was roughly equal to the number that decreased.  That is not the case here: any of the covariate shocks we consider lead to decreases in the expected number of different kinds of food acquired by the household.  This tells us that on average these shocks lead to a decrease in dietary diversity, a widely used measure of food security \autocite{hoddinott-yohannes02}.

\begin{figure}[htbp]
\centering
\includegraphics[width=.9\linewidth]{../Figures/Uganda/shocks_and_log_food_quantities.png}
\caption{\label{fig:shocks_and_log_food_quantities}Log Quantities and Shocks.  Estimated coefficients from regression of log quantities of different food items on the share of households in a given market-year reporting given covariate shock (year effects included; zeros dropped).  Bars indicate 95\% confidence intervals.  Goods are ordered according to magnitude of coefficients averaged across different shock regressions.}
\end{figure}

We next consider the intensive margin. Figure \ref{fig:shocks_and_log_food_quantities} reports results from regressions of log quantities of different kinds of food on our measure of shocks.  Here we regress log quantities (discarding zeros) on a set of time dummies and on the proportion of households in a given market-year reporting a particular shock.  A regression of a log quantity on a proportion yields a coefficient which we can interpret as an elasticity.  Foods with the largest elasticities include mangoes, different forms of cassava, and oranges.  The fact that consumption of mangoes and oranges increases in response to covariate shocks may seem surprising, but both are tree crops widely grown in Uganda and well known for their drought tolerance.  Note from \ref{fig:shocks_and_positive_food_quantities} that demand on the extensive margin responds very little to shocks, so we might guess that given that these are available in the market they may save as substitutes for other foods such as sweet bananas that are not at all tolerant of drought.   Another interesting case is ``other tobacco''; the prevalence of use increases when there's a covariate shock, but at the same time the log quantity falls.  This suggests that shocks may induce widespread adoption but modest use of this stimulant.  We've already noted the large extensive response of sim sim consumption to shocks.  However, on the intensive margin there's very little response to shocks, bolstering our interpretation that demand for sim sim is importantly related to substitution and complementarities with other foods.

One general observation about each of these figures is the fact that changes in demand for different goods tend to be highly correlated across the different covariate shocks.   Coefficients from both the level and log quantity regressions have a Spearman rank correlation coefficient of 0.92 between drought and flood shocks, for example, while the same correlation in the ``positive quantity'' regression is 0.91.   The correlation between flood and pest coefficients is systematically higher than the correlation between drought and pest, but even the latter ranges from 0.54--0.68.   We might read this as evidence that each of these shocks is operating more through its effect on a shared income shock, rather than through prices, since it seems likely that different shocks should have different effects on the local supply of different goods (this is borne out by relatively low correlations between coefficients across shocks in Figure \ref{fig:shocks_and_farmgate_prices}, which range from 0.23--0.61).

Finally, we emphasize that while the results in this section have found evidence that reported covariate shocks affect production, prices, and consumption in ways that seem consistent, that same evidence is all consistent with the hypothesis that the costs of these shocks is \emph{shared} in a fashion consistent with full-insurance.   The only evidence we've really produced for or against the full insurance hypothesis as it pertains to covariate shocks comes from the tests in Section \ref{sec:tests}.
\section{For Online Publication: Data on Shocks\hfill{}\textsc{online}}
\label{app:shocks}
Data on self-reported shocks comes from a ``shocks'' module in different rounds of the Ugandan LSMS surveys.  There are minor differences in elicitation and in the list of shocks reported in certain years.  Table \ref{shock_labels} reports the slight aggregation of shocks we use, while Table \ref{howcoped_labels} similarly indicates how we've harmonized responses to the question ``How did your household cope with this [SHOCK]?''.
\begin{table}[htbp]
\caption{\label{shock_labels}Different shock labels across eight rounds of Ugandan data, with harmonized labels.}
\centering
\begin{tabular}{p{4in}l}
Existing Label & Label\\
\hline
Conflict/Violence & Conflict\\
Death of Income Earner(s) & Death of earner\\
Death of Other Household Member(s) & Death (non-earner)\\
Drought & Drought\\
Drought/Irregular Rains & Drought\\
Erosion & Erosion\\
Fire & Fire\\
Floods & Floods\\
Irregular Rains & Drought\\
Landslides & Erosion\\
Landslides/Erosion & Erosion\\
Loss of Employment of Previously Employed Household Member(s) (Not Due to Illness or Accident) & Lost Earnings\\
Other (Specify) & Other\\
Reduction in the Earnings of Currently (Off-Farm) Employed Household Member(s) & Lost Earnings\\
Serious Illness or Accident of Income Earner(s) & Health\\
Serious Illness or Accident of Other Household Member(s) & Health\\
Theft & Theft\\
Theft of Agricultural Assets/Output (Crop or Livestock) & Theft\\
Theft of Money/Valuables/Non-Agricultural Assets & Theft\\
Unusually High Costs of Agricultural Inputs & Prices\\
Unusually High Level of Crop Pests \& Disease & Pests\\
Unusually High Level of Crop Pests \&amp; Disease & Pests\\
Unusually High Level of Livestock Disease & Pests\\
Unusually Low Prices for Agricultural Output & Prices\\
\end{tabular}
\end{table}


\begin{table}[htbp]
\caption{\label{tab:shocks_by_round}Incidence of shocks by round}
\centering
\begin{tabular}{p{2in}rrrrrrrr}
 &  &  &  &  &  &  &  & \\
Shock & 2005 & 2009 & 2010 & 2011 & 2013 & 2015 & 2018 & 2019\\
\hline
Conflict/Violence & 268 & 34 & 27 & 46 & 11 & 12 & 22 & 25\\
Death of Income Earner(s) & 99 & 27 & 17 & 19 & 30 & 19 & 20 & 19\\
Death of Other Household Member(s) & 423 & 74 & 58 & 35 & 66 & 35 & 44 & 49\\
Drought & 1234 & --- & --- & --- & 735 & 526 & 556 & 360\\
Drought/Irregular Rains & --- & 1344 & 710 & 560 & --- & --- & --- & ---\\
Erosion & --- & --- & --- & --- & 15 & 3 & 19 & 13\\
Fire & 105 & 26 & 21 & 20 & 17 & 18 & 12 & 9\\
Floods & 426 & 61 & 102 & 148 & 98 & 62 & 74 & 117\\
Irregular Rains & --- & --- & --- & --- & 179 & 72 & 180 & 169\\
Landslides & --- & --- & --- & --- & 1 & --- & 2 & 2\\
Landslides/Erosion & --- & 21 & 5 & 17 & --- & --- & --- & ---\\
Loss of Employment of Previously Employed Household Member(s) (Not Due to Illness or Accident) & --- & 9 & 10 & 8 & 5 & 10 & 5 & 6\\
Other (Specify) & 111 & 101 & 60 & 57 & 62 & 45 & 50 & 75\\
Reduction in the Earnings of Currently (Off-Farm) Employed Household Member(s) & --- & 28 & 3 & 10 & 6 & 4 & 15 & 15\\
Serious Illness or Accident of Income Earner(s) & 82 & 189 & 152 & 91 & 86 & 56 & 107 & 119\\
Serious Illness or Accident of Other Household Member(s) & --- & 188 & 149 & 65 & 47 & 32 & 83 & 78\\
Theft & 349 & --- & --- & --- & --- & --- & --- & ---\\
Theft of Agricultural Assets/Output (Crop or Livestock) & --- & 127 & 48 & 21 & 35 & 27 & 41 & 40\\
Theft of Money/Valuables/Non-Agricultural Assets & --- & 106 & 48 & 34 & 41 & 35 & 34 & 43\\
Unusually High Costs of Agricultural Inputs & 71 & 60 & 19 & 27 & 49 & 7 & 10 & 12\\
Unusually High Level of Crop Pests \& Disease & --- & 137 & 40 & 61 & 54 & 35 & --- & ---\\
Unusually High Level of Crop Pests \&amp; Disease & 292 & --- & --- & --- & --- & --- & 120 & 79\\
Unusually High Level of Livestock Disease & 183 & 82 & 37 & 31 & 17 & 18 & 10 & 5\\
Unusually Low Prices for Agricultural Output & --- & 53 & 35 & 38 & 18 & 5 & 68 & 17\\
\end{tabular}
\end{table}

\begin{table}[htbp]
\caption{\label{howcoped_labels}Different coping labels across eight rounds of Ugandan data, with harmonized labels.}
\centering
\begin{tabular}{p{4in}l}
\\
Existing Label & Label\\
\hline
Unconditional help provided by relatives/friends & Help from relatives/friends\\
Unconditional help provided by local government & Help from local government\\
Sold land/building & Sold assets\\
Sold durable household assets (agricultural or nonagricultural) & Sold assets\\
Sold durable household assets (agricultural or non-agricultural) & Sold assets\\
Sold assets & Sold assets\\
Sent children to live elsewhere & Sent children elsewhere\\
Rented out land/building & Rented out assets\\
Relied on savings & Used savings\\
Reduced expenditures on health and education & Reduced investments in human capital\\
Reduced consumption & Reduced consumption\\
Other(specify) & Other\\
Other (specify) & Other\\
Obtained credit & Used credit\\
Nothing else & Nothing\\
More wage employment & Reduced leisure\\
Increased agriculture labour supply & Reduced leisure\\
Household member(s) took on more non-farm (wage- or self-) employment & Reduced leisure\\
Household member(s) took on more farm wage employment & Reduced leisure\\
Household member(s) migrated & Migrated\\
HH member took on more non-farm (wage- or self-) employment & Reduced leisure\\
HH member took on more farm wage employment & Reduced leisure\\
Distress sales of animal stock & Sold assets\\
Did nothing & Nothing\\
Changed dietary patterns involuntarily (Relied on less preferred food options, reduced the proportion or number of meals per day, skipped days without eating, etc) & Reduced consumption\\
Changed dietary patterns involuntarily & Reduced consumption\\
Changed cropping practices (crop choices or technology) & Changed cropping practices\\
Changed cropping practices & Changed cropping practices\\
Rented out land/building & Rented out assets\\
99.0 & Nothing\\
\end{tabular}
\end{table}


\label{ref:income}
Another idiosyncratic variable we're interested in is income, if only because of how often it's used in risk-sharing tests.  There are three main sources of income for the households in these data: earnings from employment; profits from non-farm enterprises, and income from farming activities.   The last is particularly difficult to construct,\footnote{Income from farming activities would in principle include profits from crop and livestock production, but both revenues and costs are poorly measured in these surveys.  Revenues are difficult to compute because for many households a considerable amount of production is directly consumed.  Though the surveys elicit respondents' assessment of the value of this own-consumption, it would be an obvious mistake to add this to income and then test whether income is correlated with the value of consumption.  Costs are also poorly measured, and for a related reason---the on-farm labor of household members is poorly measured and difficult to value.} and so we restrict our attention to non-farm income, constructed as the sum of earnings and profits from non-farm enterprises.  In only 9724 of 22,791 total usable observations do households report any non-farm income; in the remaining cases we treat our measure of income as missing. We would not argue that this is a very good measure of income, but would argue that the full risk-sharing hypothesis implies that variation in this poor measure of income should not affect households MUEs.



\clearpage
\end{document}
